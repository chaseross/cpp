\section{Evaluation}

There is little evaluation of the FSA's 2009 stress test because the test was not publicized, its results were not released, and the FSA used the stress test as part of its ongoing supervisory approach. However, Andrew Haldane -- the head of financial stability at the Bank of England in 2009 -- described the shortcomings of the stress test that had been embedded into the normal regulatory process. \citep{Haldane}. He identifies three reasons that the stress test failed in 2008 and 2009: underestimating the likelihood of extreme events, poor understanding of spillover and linkages, and misaligned incentives. To combat these points he suggests five policies.

\paragraph{``Setting the stress scenario.''} It is important that regulators, not firms themselves, create the stress scenario as to avoid ``disaster myopia.'' Further, the test should be extreme enough that it represents a tail event.

\paragraph{``Regular evaluation of common stress scenarios.''} Basel II required banks to conduct annual stress tests, and financial institutions struggled to implement stress tests over a shorter time-frame. Regular testing would also allow direct comparisons across firms and encourage management to use the results of these tests in their normal decision process.

\paragraph{``Assessment of the second-round effects of stress.''} The results of the tests should be compared with second order effects like liquidity hoarding and fire-sales, and thereby encourage firms to think about spillover and contagions that comes with their own actions.

\paragraph{``Translation of results into firms' liquidity and capital planning.''} Results of stress tests, including the second-round effects, need to be used in management decisions and should be taken to firms' risk committees.

\paragraph{``Transparency to regulators and financial markets.''} Public disclosure of bank-specific results would impose some market discipline over management decisions. These disclosures should be standardized so that direct comparisons are possible, and the disclosures should be regularly published -- not just during moments of particular stress.

In its 2009/2010 annual report, the FSA summarized the impact of its stress test:

\begin{quote}
[The FSA] set a challenging stress test for banks based on a much more severe recession than the Bank of England’s central projection. This proactive approach resulted in banks significantly improving their capital positions which enabled them to better withstand the downturn. \citep{FSAReport}.
\end{quote}
