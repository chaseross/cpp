\section{Key Design Decisions}\label{keydesign}

\subsection{The FSA did not initially disclose details of the test and never disclosed bank-by-bank results.}

A key aspect to the FSA's stress test was the evolution of disclosure from the time the Bank Recapitalisation Scheme's announcement in October 2008 to the release of relatively more specific design information in May 2009. The plan as announced in October did not explicitly mention the FSA nor the stress test. Rather, the FSA published a press release in November 2008 which provided just three paragraphs of details on the test's method and aims, essentially declining to publicly describe the specifics of the test or when the results would be announced (if at all). The relevant paragraphs in their entirety read:

\begin{quote}
  Within the Tripartite structure, the FSA was responsible, in consultation with the other two authorities, for determining the appropriate level of capital for each individual institution. In reaching this determination two factors were taken into account:

  \begin{enumerate}
    \item Ensuring that the amount of capital for each institution would sustain confidence in that institution.
    \item Ensuring that each individual institution would have a sufficient capital buffer over minimum capital requirements both to absorb losses that might ensue from a recession and to continue lending on normal commercial criteria.
  \end{enumerate}

  To ensure broad consistency between different institutions, the process included utilisation of a stress test based on some standard assumptions but with weightings tailored to the specific institutions. The FSA used as common benchmarks within this framework ratios of capital to risk weighted assets of total Tier 1 Capital of at least 8\% and Core Tier 1 Capital, as defined by the FSA, of at least 4\% after the stressed scenario.

\end{quote}

The FSA released no comments on the test design or results, save a brief note on changes to the additional capital held as a result of Basel's procyclical effect, until May 28, 2009. Beginning in February 2009, US financial regulators had begun a stress test of the major US bank holding companies called the Supervisory Capital Assessment Program (SCAP). In April 2009 the Federal Reserve publicly published the methodology and scenarios used in the stress test, and on May 7, 2009 US regulators published bank-by-bank details describing exposures by asset class along with estimates of bank-specific capital shortfalls. It was dramatic departure from the disclosure precedent by the FSA.

In the period between formal announcement of the stress test and May 2009, the FSA provided little information on the stress test. After Lloyds and RBS were recapitalized with public funds in October, attention turned to Barclays as the last large bank which might require substantial public support. However, this turned out to not be the case when the \textit{Financial Times} learned that the FSA had concluded its stress test of Barclays in April 2009 with no additional capital injection. \citep{Barclays1} and \citep{Lex}.

In light of the opaque announcement regarding Barclays, some market participants felt that SCAP-like disclosures would bolster both confidence in the banking system but also in the FSA:

\begin{quote}
  More than ever, however, the FSA’s credibility is on the line. By giving the bank the all-clear, it has indicated that it is satisfied the group will be able to maintain core tier one capital of 4 per cent at the trough of the recession, the minimum requirement it stipulated in January... The FSA must address concerns over the rigour of its stress testing by detailing its worst-case assumptions. Is it sufficient to test Barclays against merely a replay of an early 1980s-type fall in employment and gross domestic product, for example? And does the FSA recognise the extent to which UK house prices remain hideously overvalued? A pass is worth little if the examiners are complicit in rampant grade inflation. \citep{Lex}.
\end{quote}

Pressure built for the FSA to release additional details after the May 7 SCAP disclosures in the United States, and Bloomberg submitted a Freedom of Information request to HM Treasury for the results of the stress test. HM Treasury denied the request, saying disclosure of the results ``at this time may lead to uncertainty in financial markets, either in relation to specific institutions or more generally... [s]uch instability could require further action by the authorities.'' UK regulators were likely reluctant to release the details of the stress test because it would be difficult to explain the differences between the scenarios used in the SCAP and the FSA tests, and also because markets may mistake the scenarios as official forecasts -- a particularly scary scenario given that rumors suggested the FSA was testing with a 50\% decline in home prices and a 16\% fall in GDP.\footnote{GDP ultimately dropped 4.2\% in the US peak to trough; 4.9\% in the UK.} \citep{Murphy}.

On May 28, the FSA released a detailed methodology statement, although the disclosures did not match the SCAP's level of detail. The release directly noted the impact SCAP had on the FSA's decision to release results, ``the publication in the US of the results of bank stress tests [provoked] substantial interest in the use of stress testing in other countries'' and that the two programs were not comparable: ``The UK authorities have not applied stress testing in the same way as in the US.'' \citep{Results}.

\subsection{The stress test was not a one-off program.}

The announcement emphasized the FSA's use of stress testing as a part of its ongoing supervisory regime. Rather than use the test as a one-off program, the FSA outlined its use of stress testing in its ongoing regulatory approach:

\begin{enumerate}
  \item ``Greatly increased the use of stress tests as an integral element of our ongoing supervisory approach.''
  \item ``Begun the process of embedding this revised approach in our intensive supervisory regime.''
  \item ``Used stress tests to inform policy decisions such as access to the Credit Guarantee Scheme (CGS) and the Asset Protection Scheme (APS) working closely with the other Tripartite authorities.'' \citep{Results}.
\end{enumerate}

This approach contrasted with contemporaneous tests in the US and EU which were both publicized as a one-off test. Accordingly, the FSA note ``[s]tress testing can and has been used in a variety of different ways, and the appropriate degree of disclosure varies according to the purposes of the tests\dots Since the FSA’s use of stress tests has not been a one-off exercise, but instead embedded in our regular supervisory processes, the FSA will not, as a matter of practice, be publishing details of the stress test results.'' \citep{Results}.