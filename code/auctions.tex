\documentclass[8pt]{beamer}
%\usetheme{CambridgeUS}
%\usecolortheme{dove}
\setbeamertemplate{caption}[numbered]
\setbeamertemplate{footline}[frame number]
\setbeamerfont{caption}{size=\scriptsize}
\usepackage[gen]{eurosym}
\usepackage{float}
\usepackage{enumitem}
\setlist[itemize]{leftmargin=*}
\usepackage{moresize}
\pdfmapfile{+sansmathaccent.map}

\usepackage{tikz}
\usepackage{pgfpages}
\setbeamertemplate{background canvas}{
    \tikz \draw (current page.north west) rectangle (current page.south east);
        }
\pgfpagesuselayout{2 on 1}[letterpaper,border shrink=5mm]
\beamertemplatenavigationsymbolsempty
\usepackage{caption}
%\setbeameroption{show notes} %un-comment to see the notes
\setbeamerfont{note page}{size=\ssmall}
\graphicspath{{../input/}}
\usepackage[outdir=./]{epstopdf}


\begin{document}

\DeclareGraphicsExtensions{.eps}

\title{Capital Purchase Program Auctions}
%\author{Chase Ross}
%\date{March 23, 2016}

%\beamersetaveragebackground{black}
%\begin{frame}
%\frametitle{}

%\end{frame}
%\beamersetaveragebackground{white}

%\frame{\titlepage}

%\frame{\frametitle{Table of contents}\tableofcontents}

\frame{\frametitle{Capital Purchase Program Auctions}

\vspace*{\fill}
\begin{figure}[h]
\noindent
\makebox[\textwidth]{\includegraphics[scale=.7 ]{Exits.pdf}}%

\end{figure}
\textbf{Background}
\begin{itemize}[label={--}]
\item Under CPP Treasury invested \$205 billion in 707 financial institutions between October 2008 and December 2009. 
\item CPP securities came with 5\% dividend for first 5 years, 9\% after. Warrants were also involved.
\item May 2012: Treasury announced strategy to wind-down CPP portfolio which emphasized auctions. 
\item Exit from CPP possible through: (1) repayment, (2) restructuring w/ new capital, or (3) auction.
\end{itemize}
}

\frame{\frametitle{ }
\begin{columns}[T] % align columns
\begin{column}{.49\textwidth}

\textbf{Design}
\vspace{2mm}

\begin{itemize}[label={--}]
\item Auctions sold either a single institution's CPP securities or pooled many firms' securities together depending on size.
\item The bank (with their regulators' approval) or a designate bidder (a familiar shareholder) could submit an ``opt-out bid'' to be removed from the set of firms to be auctioned. 
\item Modified Dutch Auction:  price of securities lowered until there are enough bids to sell all the securities and all the securities are then sold at that price.
\item ``Modified'' in the sense that there was a floor, often set by the firm's opt-out bid.
\item In pooled auctions, a single bidder is allocated all auctioned securities.
\item In single institution auctions, many bidders are allocated portions of the auctioned securities at the single clearing price.
\item Treasury previously used Dutch auctions to sell CPP warrants. 
\end{itemize}

\end{column}%
\hfill%
\begin{column}{.49\textwidth}

\textbf{Outcomes, as of June 10, 2016}
\vspace{2mm}

\begin{itemize}[label={--}]
\item Of \$205 billion disbursed, \$199.6 repaid, \$5.1 billion written off, \$300 million outstanding, \$27.1 billion in returns. 
\item 28 auctions for 190 CPP institutions yielding \$3.04 billion in proceeds.
\item Auctions returned about 80\% face value.
\item 14 institutions remain; expected to exit via (mostly) restructuring.
 \end{itemize}

\textbf{Evaluation}
\vspace{2mm}
\begin{itemize}[label={--}]
\item Opt-out bid = floor price:  some questioned the use of the firm's opt-out bid as the auction floor; Treasury did not explicitly say they'd do this. 
\item Disclosure requirements: publicly traded banks complained the SEC required them to publicly disclose their intent to bid and the amount of capital raised to do so.
\item No matching bids: many banks wanted the ability to match the winner's bid, but Treasury didn't allow it.
\end{itemize}
\end{column}%
\end{columns}
}
\end{document}
