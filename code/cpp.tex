\documentclass[12pt]{article}
\usepackage{longtable}
\usepackage{geometry}
\geometry{left=1.5 in,right=1.5 in,top=1.00in,bottom=1.0in}

%\documentclass[justified,notoc]{tufte-handout2}
%need to figure out why figure refernces are messed up

\usepackage{natbib}  % call natbib
\setcitestyle{authoryear} % set citation style to authoryear
\bibliographystyle{plainnat} % use the plainnat instead of plain
\usepackage{booktabs}
\usepackage{amssymb, amsmath, amsfonts}
\usepackage{outlines}
\usepackage{soul}
\usepackage{textcomp}
\usepackage[gen]{eurosym}
\usepackage{graphicx}
\usepackage{enumitem}
\usepackage{lineno}
\setlist[enumerate]{itemsep=0mm}
\setlist[itemize]{itemsep=0mm}
\usepackage{fancyhdr}
\usepackage{authblk}
\usepackage{lineno}
\usepackage[hyphens]{url}
\usepackage{hyperref}
\hypersetup{
 colorlinks = true, %Colours links instead of ugly boxes
 urlcolor = blue, %Colour for external hyperlinks
 linkcolor = blue, %Colour of internal links
 citecolor = blue %Colour of citations
}
\pagestyle{fancy}
\usepackage{outlines}
\usepackage{caption}
%\captionsetup[table]{name=Figure}
\graphicspath{{../input/}}
\usepackage[outdir=./]{epstopdf}


\usepackage{enumitem}
\setlist[enumerate,2]{label=\roman*)}
\setlist[enumerate,3]{label=\Roman*)}
\setlist[enumerate,4]{label=\roman*)}

%\renewcommand{\familydefault}{\sfdefault}
% ------------------------------- use \citep{FAQs} to cite
\usepackage{filecontents}
\begin{filecontents}{\jobname.bib}
}

@article{Ross2016a,
 title  = {{The Supervisory Capital Assessment Program}},
 publisher = "Yale Program on Financial Stability",
 author = "Ross, Chase P.",
 year  = "2016",
 doi  = " ",
 volume = " ",
 journal = "Yale Program on Financial Stability Intervention Case",
 issn  = " ",
 url  = {http://papers.ssrn.com/sol3/papers.cfm?abstract_id=2722712},
}

@article{Ross2016b,
 title  = {{The Capital Assistance Program}},
 publisher = "Yale Program on Financial Stability",
 author = "Ross, Chase P.",
 year  = "2016",
 doi  = " ",
 volume = " ",
 journal = "Yale Program on Financial Stability Intervention Case",
 issn  = " ",
}


@book{Geithner,
 title={{Stress Test: Reflections on Financial Crises}},
 author={Geithner, Timothy F},
 year={2014},
 publisher={Crown}
}

@misc{Paulson,
 title={{October 14, 2008 Speech}},
 author={Paulson, Hank},
 year={2008},
  url  = {http://ftalphaville.ft.com/2008/10/14/17036/paulsons-nine-strong-posse/},
}

@misc{CPPInstructions,
 title={{Application Guidelines for TARP Capital Purchase Program}},
 author={Treasury},
 year={2008},
  url  = {https://www.treasury.gov/initiatives/financial-stability/TARP-Programs/bank-investment-programs/cap/Documents/application-guidelines.pdf 
},
}

@book{Bernanke,
 title={{The Courage to Act: A Memoir of a Crisis and its Aftermath}},
 author={Bernanke, Ben S},
 year={2015},
 publisher={WW Norton \& Company}
}

@article{CAPTerms,
 title  = {{U.S. Treasury Releases Terms of Capital Assistance Program}},
 author = "U.S. Treasury",
 year  = "2009",
 doi  = " ",
 volume = " ",
 journal = " ",
 issn  = " ",
 url  = {https://www.treasury.gov/press-center/press-releases/Pages/tg40.aspx},
}


@article{Mofo,
 title={{Capital Assistance Program Public (CAP) Cheat Sheet}},
 author={{Morrison Foerster}},
 year={2009},
url  = {http://media.mofo.com/docs/pdf/090310CAPPublicCheatSheet.pdf},
}

@article{OFS,
 title={{Troubled Asset Relief Program: Two Year Retrospective}},
 author={{Office of Financial Stability}},
 year={2010},
url  = {https://www.treasury.gov/press-center/news/Documents/TARP\%20Two\%20Year\%20Retrospective_10\%2005\%2010_transmittal\%20letter.pdf},
}


@article{CPPFaq,
 title  = {{Process-Related FAQs for Capital Purchase Program}},
 author = "U.S. Treasury",
 year  = "2008",
 doi  = " ",
 volume = " ",
 journal = " ",
 issn  = " ",
 url  = {https://www.treasury.gov/press-center/press-releases/Documents/faqcpp.pdf},
}


@article{Andrews,
 title={{U.S. May Convert Banks' Bailouts to Equity Share}},
 author={Andrews, Edmund L.},
 year={2009},
 publisher={The New York Times},
 url={http://www.nytimes.com/2009/04/20/business/20bailout.html?_r=1},
}


@article{kacperczyk,
 title={When Safe Proved Risky: Commercial Paper during the Financial Crisis of 2007--2009},
 author={Kacperczyk, Marcin and Schnabl, Philipp},
 journal={The Journal of economic perspectives},
 volume={24},
 number={1},
 pages={29--50},
 year={2010},
 publisher={American Economic Association}
}

@article{Khan,
 title={The Capital Purchase Program and Subsequent Bank SEOs},
 author={Khan, Mozaffar and Vyas, Dushyantkumar},
 journal={Journal of Financial Stability},
 volume={18},
 pages={91--105},
 year={2015},
 publisher={Elsevier}
}


@book{paulsonbook,
 title={On the Brink: Inside the Race to Stop the Collapse of the Global Financial System},
 author={Paulson, Henry},
 year={2010},
 publisher={Business Plus}
}

@article{Greenlaw,
 title={Leveraged Losses: Lessons from the Mortgage Market Meltdown},
 author={Greenlaw, David and Hatzius, Jan and Kashyap, Anil K and Shin, Hyun Song},
  journal={U.S. Monetary Policy Form 2008}
}

@article{IMF2010,
 title={Funding, and Systemic Liquidity},
 author={Sovereigns, IMF},
 journal={Global Financial Stability Report},
 year={2007} 
}

@book{Elliott,
 title={A Primer on Bank Capital},
 author={Elliott, Douglas J},
 year={2010},
 publisher={Brookings Institution}
}

@book{Sorkin,
 title={Too Big to Fail},
 author={Sorkin, Andrew Ross},
 year={2009},
 publisher={Viking}
}

@article{Peek,
 title={The International Transmission of Financial Shocks: The Case of Japan},
 author={Peek, Joe and Rosengren, Eric S},
 journal={American Economic Review},
 volume={87},
 number={4},
 year={1997}
}

@article{Hoshi,
 title={Will the U.S. Bank Recapitalization Succeed? Eight Lessons from Japan},
 author={Hoshi, Takeo and Kashyap, Anil K},
 journal={Journal of Financial Economics},
 volume={97},
 number={3},
 pages={398--417},
 year={2010},
 publisher={Elsevier}
}

@misc{shortban,
 title={{SEC Halts Short Selling of Financial Stocks to Protect Investors and Markets}},
 author={{Securities Exchange Commission}},
 year={2008},
  url  = {https://www.sec.gov/news/press/2008/2008-211.htm},
}

@misc{MUFGMS,
 title={{MUFG to Renegotiate Morgan Stanley Deal}},
 author={Robinson, Gwen},
 year={2008},
  publisher={The Financial Times},	
  url  = {http://ftalphaville.ft.com/2008/10/13/16948/mufg-to-renegotiate-morgan-stanley-deal/},
}

@misc{UKPlans,
 title={{UK to Inject \textsterling 39bn into Banks}},
 author={Larsen, Peter},
 year={2008},
  publisher={The Financial Times},	
  url  = {http://www.ft.com/cms/s/0/153e175e-9883-11dd-ace3-000077b07658.html},
}

@misc{UKActual,
 title={{UK Launches \textsterling 37bn Bank Rescue}},
 author={Robinson, Gwen},
 year={2008},
  publisher={The Financial Times},	
  url  = {http://www.ft.com/cms/s/0/83bc2cea-98ef-11dd-9d48-000077b07658.html},
}

@misc{GSBHC,
 title={{Goldman Sachs to Become the Fourth Largest Bank Holding Company}},
 author={{Goldman Sachs}},
 year={2008},
  url  = {http://www.goldmansachs.com/media-relations/press-releases/archived/2008/bank-holding-co.html},
}

@misc{mofo,
 title={{Capital Purchase Program Cheat Sheet}},
 author={{Morrison Foerster}},
 year={2008},
  url  = {http://media.mofo.com/docs/pdf/081031CaPPCheatSheet_Private.pdf},
}

@article{CPPTerms,
 title  = {{TARP Capital Purchase Program, Senior Preferred Stock and Warrants. Summary of Senior Preferred Terms}},
 author = "U.S. Treasury",
 year  = "2008",
 doi  = " ",
 volume = " ",
 journal = " ",
 issn  = " ",
 url  = {https://www.treasury.gov/press-center/press-releases/Documents/document5hp1207.pdf},
}

@misc{CPPAnnouncement,
 title={{Treasury Announces TARP Capital Purchase Program Description}},
 author={{U.S. Treasury}},
 year={2008},
  url  = {https://www.treasury.gov/press-center/press-releases/Pages/hp1207.aspx},
}

@misc{mofo2,
 title={{Update to Treasury's Capital Purchase Program}},
 author={{Morrison Foerster}},
 year={2008},
  url  = {http://www.mofo.com/~/media/Files/Resources/Publications/2008/10/Update\%20to\%20Treasurys\%20Capital\%20Purchase\%20Program/Files/081021TreasuryUpdate/FileAttachment/081021TreasuryUpdate.pdf},
}

@misc{FedTier1,
 title={{Press Release: Regarding CPP Shares and Tier 1 Capital}},
 author={{Board of Governors of the Federal Reserve System}},
 year={2008},
  url  = {https://www.federalreserve.gov/newsevents/press/bcreg/20081016b.htm},
}

@article{RejectionRates,
 title={How Do Banks Use Bailout  Money? Optimal Capital Structure, New Equity, and the TARP},
 author={Taliaferro, Ryan},
 journal={Optimal Capital Structure, New Equity, and the TARP (December 21, 2009)},
 year={2009}
}

@misc{mofoComp,
 title={{TARP Executive Compensation}},
 author={{Morrison Foerster}},
 year={2009},
  url  = {http://media.mofo.com/docs/pdf/090310ExecutiveCompCheatSheet.pdf},
}

@misc{mofoCap,
 title={{Capital Alternatives for Financial Institutions: Treasury's TARP Capital Purchase Program}},
 author={{Morrison Foerster}},
 year={2009},
  url  = {http://media.mofo.com/docs/pdf/0903CaPPOverview.pdf},
}


@misc{MassadExit,
 title={{Winding Down TARP's Bank Programs}},
 author={Massad, Timothy},
 year={2012},
  url  = {https://www.treasury.gov/connect/blog/Pages/Winding-Down-TARPs-Bank-Programs.aspx},
}

@article{GAO,
 title={Troubled Asset Relief Program: Capital Purchase Program Largely Has Wound Down},
 author={{Government Accountability Office}},
 year={2016},
 url = {http://www.gao.gov/products/GAO-16-524}
}

@misc{ProgramStatus,
 title={{Capital Purchase Program, Program Status}},
 author={{U.S. Treasury}},
 year={2016},
  url  = {https://www.treasury.gov/initiatives/financial-stability/TARP-Programs/bank-investment-programs/cap/Pages/payments.aspx},
}

@article{Ba,
 title={Assessing TARP},
 author={Bayazitova, Dinara and Shivdasani, Anil},
 journal={Review of Financial Studies},
 volume={25},
 number={2},
 pages={377--407},
 year={2012},
 publisher={Soc Financial Studies}
}

@misc{SmallRepay,
 title={{Four Small Banks Are the First to Pay Back TARP Funds}},
 author={Dash, Eric},
 year={2009},
  publisher={The New York Times},	
  url  = {http://nyti.ms/29aFjv3},
}

@misc{TIP,
 title={{Targeted Investment Program, Program Purpose and Overview}},
 author={{U.S. Treasury}},
 year={2008},
  url  = {https://www.treasury.gov/initiatives/financial-stability/TARP-Programs/bank-investment-programs/tip/Pages/default.aspx},
}

@article{Diamond,
 title={Fear of Fire Sales and the Credit Freeze},
 author={Diamond, Douglas W and Rajan, Raghuram G},
 year={2009},
 institution={National Bureau of Economic Research}
}

@article{Gift,
 title={Paulson's Gift},
 author={Veronesi, Pietro and Zingales, Luigi},
 journal={Journal of Financial Economics},
 volume={97},
 number={3},
 pages={339--368},
 year={2010},
 publisher={Elsevier}
}

@article{GW,
 title={Valuing the Treasury's Capital Assistance Program},
 author={Glasserman, Paul and Wang, Zhenyu},
 journal={Management Science},
 volume={57},
 number={7},
 pages={1195--1211},
 year={2011},
 publisher={INFORMS}
}

\end{filecontents}
% -------------------------------

\begin{document}

	\lhead{}
	\rhead{}
	\renewcommand{\headrulewidth}{0.0pt}
	\renewcommand{\footrulewidth}{0.0pt}

\title{The Capital Purchase Program}
\author{Chase P. Ross\thanks{\texttt{\href{mailto:chase.ross@yale.edu}{chase.ross@yale.edu}}}}
\affil{Yale Program on Financial Stability \\ Yale School of Management}
\date{}


\maketitle

\begin{abstract}
On October 14, 2008 U.S. policymakers announced a plan to purchase preferred shares in stressed U.S. banks and financial institutions to ensure the U.S. banking system had sufficient capital to withstand further economic deterioration. Using \$250 billion in capital from the Toxic Asset Relief Program (TARP), policymakers used preferred shares to inject \$205 billion into 707 U.S. financial institutions through the Capital Purchase Program (CPP). As of June 2016, the Treasury had received \$226.7 billion in repayments, dividends, interest and other income associated with the program.
\newline
\newline
\textbf{JEL Classification}: G01, G28, G20, H12, H81
\newline
\textbf{Keywords}: crisis intervention, Capital Purchase Program, TARP, capital requirements, systemic risk, Tier 1 Capital

\end{abstract}
\newpage
\tableofcontents
\newpage
\linenumbers
\section{Overview}

\subsection{Background}

Financial markets became increasing volatile beginning in the summer of 2007 and sharply more so following the bankruptcy of Lehman brothers on September 15, 2008 and the subsequent runs on AIG and the Reserve Primary Fund. Combined, the run on the securitized banking system escalated in September 2008 and financial institutions hoarded cash as uncertainty about the value of banks' assets pushed interbank funding rates up. Estimates at the time \$500 billion in losses within the mortgage market through 2008. \citep{Greenlaw}.\footnote{The IMF estimated about \$700 billion in aggregate losses and write-downs within the U.S. banking system between 2007Q2 and 2010 Q2. \citep{IMF2010}.} The provision of private credit collapsed. Emergency liquidity provision by the Federal Reserve helped viable firms finance themselves, but policymakers and market analysts were concerned banks did not have sufficient capital to absorb additional losses. As a response, Congress passed the Emergency Economic Stabilization Act of 2008 (EESA) in early October 2008.\footnote{Congress initially tried to pass the law, with a handful of differences, in late September but the bill did not pass. The S\&P lost 8.81 percent that day.} The EESA created the Toxic Asset Relief Program (TARP) with \$700 billion in funding.

Initially, policymakers designed TARP to purchase impaired assets from qualified regulated financial institutions. These purchases would prevent banks from selling their assets at fire-sale prices and using their limited capital buffers to absorb the losses. However, policymakers at the Treasury and Federal Reserve ultimately decided it was best to purchase equity directly from banks instead for three reasons: first, policymakers were concerned the process of setting up the purchasing program's logistics would take too long\footnote{The Treasury expected it would take 45 days until the program could begin its purchases. \citep{Geithner}.}; second, it was difficult to set appropriate prices for securitized assets with no market price\footnote{For example, the Federal Reserve and JPMorgan struggled for many months to negotiate appropriate prices for the pool of Bear Stearns mortgage securities JPMorgan agreed to purchase in March 2008 as part of the Maiden Lane I transaction. \citep{Geithner}.}; and third, equity was a more efficient use of the limited TARP funds than asset purchases.\footnote{For a bank with ten to one leverage, \$1 billion in equity investments is equal to \$10 billion in asset purchases.}

Capital adequacy is vital for banks to intermediate credit. \citet{Peek} shows that banks without sufficient common equity pull back from lending. When supervisors and banks are unable or unwilling to recapitalize the banking system, outcomes for a wide variety of macroeconomic indicators are dim: investment, output and unemployment all suffer. \citep{Hoshi}. Further, in the week leading up to the announcement of capital injections, banks struggled to finance their operations while interbank funding rates increased to all-time highs: the TED spread\footnote{The TED spread is the spread between short-term U.S. Treasury bills (T) and eurodollar futures (ED).} peaked at an all-time high of 458 basis points on October 10, 2008 as shown in Figure~\ref{ted}. Equity markets reflected the stresses, as well: the week of October 6 was the worst week for the S\& P since 1933.\footnote{As an indication of the unprecedented level of uncertainty in financial markets, on Friday October 10 the Dow fell 680 points, rebounded 631 points, wiped out the rebound, then rebounded yet again 853 points but then ultimately gave up 129 points on the day. Similarly, the S\&P lost 8 percent in the last hour of trading alone. \citep{paulsonbook}.} 

The week of October 6 was particularly stressful for the U.S. banking system. Notably, the SEC had banned short selling of financial stocks beginning September 19 and expiring on Wednesday October 8. \citep{shortban}. At the same time, Morgan Stanley struggled to remain viable as it was seen as the most vulnerable bank following Lehman. On September 29, Mitsubishi UFJ agreed to purchase 9 percent of Morgan Stanley at \$25.25, with the deal closing on October 13. Morgan Stanley executives expressed frustration that the short-sale ban would expire just days before the deal closed. After the ban expired, Morgan Stanley's shares fell more than 60 percent and market analysts remained uncertainty whether the deal would indeed the following week. Ultimately, Mitsubishi renegotiated the deal to purchase a 21 percent stake for \$9 billion.\footnote{Treasury Secretary Paulson, concerned Mitsubishi may pull back from the deal, sent the board of Mitsubishi a letter the earlier in the day which outlined the principles of their policy actions and that they intended to protect foreign investors, but did not explicitly mention the deal or Morgan Stanley. Within a few hours, Mitsubishi had agreed to the finalized deal. \citep{paulsonbook}.}\footnote{As the deal closed on Columbus day -- a bank holiday -- Mitsubishi was unable to wire the \$9 billion check. Morgan Stanley needed the cash on an emergency basis, so Mitsubishi gave Morgan Stanley a physical check for the \$9 billion. The check is likely the largest ever written. \citep{Sorkin}.} \citep{MUFGMS}. Further, AIG had depleted its \$85 billion bridge loan the same week and U.S. policymakers had to set up an additional \$37.8 billion program. \citep{Geithner}. 

The U.K. government set the precedent for direct capital injections on October 8, 2008 when U.K. policymakers unveiled a \textsterling 400 billion bank assistance package which included \textsterling 25 billion to recapitalize the banking system, with an additional \textsterling 25 billion available if policymakers deemed necessary. The recapitalization plan also included the creation of a guarantee scheme of \textsterling 250 billion for new wholesale debt from banks with a plan to boost Tier 1 capital in the amount supervisors deemed appropriate. U.K. banks had until the end of the year to submit capital plans and their plans to raise private capital. \citep{UKPlans}. However, the panicky selling leading into the weekend of October 11 compelled U.K. policymakers to speed up the recapitalization process so they could announce the terms of capital injections by market open on Monday October 13. The banks, Barclays in particular, sought extra time to raise private capital and therefore avoid having the government as their largest shareholder. Shortly thereafter U.K. regulators announced a \textsterling 9 billion investment in RBS and Lloyds/HBOS preferred shares, giving the government a majority share of RBS and more than 40 percent share of Lloyds/HBOS. Barclays agreed to forgo its dividends through 2008 while it sought to raise an additional \textsterling 10 billion privately. \citep{UKActual}.

Shortly thereafter, U.S. policymakers announced direct capital injections with TARP funds on October 14, 2008 through the so-called Capital Purchase Program (CPP) with \$250 billion from TARP funds. 

\subsection{Program Description}

This case divides the CPP into three distinct phases: the initial round of capital injection on October 13, Columbus Day, to 9 of the largest bank holding companies (BHCs), the application-based CPP investments available to other, mostly smaller, banks between October 2008 and November 2008, and Treasury's subsequent exit from its CPP investments which continues as of July 2016. 

\subsection*{Columbus Day Capital Injections}

Following the tumultuous week of October 6 -- the worst week for the S\&P since 1933, as well as notable stresses on Morgan Stanley and AIG -- regulators decided they needed to do something ``dramatic.'' \citep{paulsonbook}. Policymakers had little time to develop a plan. In addition to developing a new wholesale bank debt guarantee program, former President of the New York Federal Reserve Timothy F. Geithner noted ``Treasury and Fed officials'] other challenge that weekend was to figure out the structure of capital investments--what kind of capital, who would get it, how to price it, and so forth. This would be the most sweeping government intervention in the private markets since the 1930s, and we had two days to design it.'' \citep{Geithner}. The plan they ultimately developed was similar to one used in the UK, capital injections through preferred shares alongside a guarantee\footnote{This guarantee was the Debt Guarantee Program (DGP), itself one of two components of the Temporary Liquidity Guarantee Program (TGLP). The second component was the Transaction Account Guarantee Program (TAGP). See the Yale Program on Financial Stability 's paper on the program: [tk]. See also: \url{https://www.fdic.gov/regulations/resources/tlgp/}.} on new wholesale bank debt; in fact, U.S. policymakers had received a copy of the UK's capital plan beforehand and found the terms to be more punitive than the U.S. plan called for. \citep{paulsonbook}. 

Policymakers feared stigma surrounding the CPP would prevent the firms which needed capital from using the program. The U.K. program was voluntary and only the weaker firms accepted capital -- Barclays, HSBC, and Standard Chartered refrained from participation -- and markets punished those weak firms. However, no U.S. regulator had the ability to compel private financial institutions to accept government capital. To address this, the CPP came with a 5 percent dividend for the first five years which regulators hoped was cheap enough to get all nine banks to participate. Further, to persuade the most important banks to join the program and therefore prevent the program from being stigmatized, policymakers arranged a meeting between the heads of the relevant regulatory agencies with the leaders of the most systemically important banks. 

Hank Paulson, the U.S. Treasury Secretary, invited the leaders of the major nine U.S. banks\footnote{These banks included JP Morgan, Bank of America, Citigroup, Wells Fargo, Goldman Sachs, Morgan Stanley, Merrill Lynch (although Bank of America was acquiring it), State Street and Bank of New York Mellon.} to meet on Monday October 13. Treasury did not select the banks invited to this initial CPP meeting; consistent with the final design of the CPP which called for each banks' application to be approved and overseen by their relevant federal banking regulator (FBR), the New York Fed and the OCC selected the banks. Regulators chose the banks based on their systemic importance rather than business line -- the banks included the four largest commercial banks, three investment banks\footnote{Bank of America had agreed to purchase Merrill Lynch on September 15, 2008 with the transaction closing in the first quarter of 2009. Additionally, both Morgan Stanley and Goldman Sachs converted from independent investment banks to bank holding companies on September 21. The two converted to BHCs for similar reasons: the ``market views oversight by the Federal Reserve and the ability to source insured bank deposits as providing a greater degree of safety and soundness... [Goldman Sachs] view[s] regulation by the Federal Reserve Board as appropriate and in the best interest of protecting and growing our franchise across our diverse range of businesses.'' \citep{GSBHC}.} and two clearing and settlement banks with systemic importance to underlying financial infrastructure. 

Some bank leaders were reluctant to take the capital, but ultimately agreed to take the capital -- even if they initially felt it was unnecessary. The CPP funds and the FDIC guarantee of new wholesale bank debt were a joint package, banks could not choose one or the other. \citep{Geithner}. At its core, the program depended on whether the strongest and most systematically important banks would join the program, and by calling the leaders of these banks together regulators were able to convince them of the value of participation. Table~\ref{columbusDay} lists the amount of capital investment in each of the nine banks as determined on Columbus Day. 

A day before the CPP's announcement on October 14, Assistant Secretary Neel Kashkari announced that Treasury would pursue a standardized program to purchase equity across a wide set of financial institutions. \citep{Ba}. Thus, leaks emerged about the nature of the meeting and markets responded positively -- the S\&P had its largest one day point increase to date. The program was formally announced the next day on October 14, 2008. The press statement announcing the program noted, ``Nine large financial institutions already have agreed to participate in this program, moving quickly and collectively to signal the importance of the program for the system. These healthy institutions have voluntarily agreed to participate on the same terms that will be available to small and medium-sized banks and thrifts across the nation.'' \citep{CPPAnnouncement}. In sum, Treasury allocated \$250 billion from TARP to the CPP, with \$125 billion allocated to the first nine banks and the remainder available through the application-based program.

The following summarizes the key features of the CPP program, which are discussed in detail below:\footnote{See \citet{mofo}.}

\begin{itemize}[label={--}]
\item Preferred investment of 1 to 3 percent of risk-weighted assets (RWA) to qualified financial institutions (QFIs).
\item 5 percent dividend for first 5 years, 9 percent after.
\item QFIs included U.S. BHCs and banks; excluded foreign institutions or U.S. branches or agencies of foreign firms.
\item Included 10-year warrants with option for Treasury to purchase amount equal to 15 percent of preferred equity.
\item Various compensation and management restrictions (increased in February 2009).
\item To exit preferred equity must be redeemed in full with ``qualified equity offering'' with regulators' approval
\item After repayment of preferred, firms could purchase back warrants at fair market price.
\item If redeemed before January 2010 firms would get a discount on the warrants.
\end{itemize}

\paragraph{Eligibility}

Qualifying financial institutions (QFIs) were eligible to apply to the CAP. QFIs included BHCs, financial holding companies, insured depository institutions, and savings and loan holding companies, that were organized and operating in the United States, and deemed viable by the appropriate federal banking agency. Financial institutions controlled by foreign entities were ineligible. S corporations and mutual depository institutions were ineligible, but were eligible for another program. Public firms electing to participate must have submitted an application to their primary federal banking regulator before November 14, 2008. \citep{mofo} and \citep{CPPTerms}. The Treasury noted the program was not ``first come first served'' in that Treasury had sufficient capital to allocate from TARP for all QFIs which chose to participate. 

\paragraph{Preferred Terms}

The CPP purchased preferred shares from qualified financial institutions (QFIs) in the amount of 1 to 3 percent of risk-weighted assets or \$25 billion, whichever was less. The CPP shares were non-voting senior preferred shares except ``on matters that could adversely affect the shares.'' \citep{CPPAnnouncement}. The shares carried a 5 percent dividend for 5 years and then increased to 9 percent thereafter. The preferred shares would count towards the firm's Tier 1 capital\footnote{The Federal Reserve's interim final rule from October 17, 2008 allowed without restriction the use of CPP preferred shares within Tier 1 capital. \citep{FedTier1}.}, and was senior to common equity and pari passu with existing preferred shares. The CPP shares were callable after 3 years and Treasury was able, at any time, to transfer the shares to a third party. 

Dividends were due quarterly, and if a participating institution did not pay dividends in full for more than 6 quarterly periods Treasury could appoint two directors. Once the institution paid dividends for 4 consecutive quarters Treasury would remove its directors. The shares had a liquidation preference of \$1,000 per share, or higher if necessary given the authorized preferred stock. Additionally, QFIs with CPP shares were not allowed to issue dividends for the 3 years following the CPP investment. The CPP also required Treasury's consent before any share repurchases other than would normally occur or for benefits. In the case that QFIs had existing covenants or other limitations on issuance -- for example, anti-dilution protections on existing preferred shares -- firms had 30 days after approval of the program to handle any necessary corporate actions. \citep{mofo2}.

\paragraph{Warrants}

In addition to Treasury's purchase of preferred shares, Treasury would also receive 10-year warrants equal to 15 percent of the preferred CPP investment. The warrants would come with a strike price equal to the average price of the firms' common stock in the 20 days preceding the issuance of the CPP preferred shares. In the case shareholders withhold consent on the warrants the warrant's strike price would decrease 15 percent per 6 months up to a maximum 45 percent discount. \citep{CPPAnnouncement}. Treasury did not receive warrants if the CPP investment was less than \$50 million or if the QFI was a certified Community Development Financial Institution. \citep{mofo}. 

\paragraph{Redemption}

Participating CPP institutions must redeem the preferred CPP shares with a qualified equity offering of any Tier 1 perpetual preferred or common stock. \citep{CPPAnnouncement}. A qualifying equity offering is the sale of Tier 1 qualifying perpetual preferred or common stock for cash. Within the first 3 years of the CPP investment the proceeds of the offering be at least 25 percent or more of the CPP preferred investment, and after 3 years can be any amount at any time. Consent of the primary federal banking regulator is required before any qualifying equity offering. Repayment If the QFI redeemed its preferred shares before December 31, 2009, Treasury reduced the number of common shares associated with its warrants in half. The QFI could purchase any other equity securities or warrants held by Treasury at fair value after the QFI had redeemed its CPP preferred shares in full. \citep{CPPTerms} and \citep{mofo2}. 

\paragraph{Executive Compensation Restrictions}

The CPP as announced four main compensation, requiring financial institutions (quoted from \citet{CPPAnnouncement}):
\begin{enumerate}
\item `` ensur[e] that incentive compensation for senior executives does not encourage unnecessary and excessive risks that threaten the value of the financial institution,''
\item ``required clawback of any bonus or incentive compensation paid to a senior executive based on statements of earnings, gains or other criteria that are later proven to be materially inaccurate,''
\item ``prohibition on the financial institution from making any golden parachute payment to a senior executive based on the Internal Revenue Code provision,''
\item ``agreement not to deduct for tax purposes executive compensation in excess of \$500,000 for each senior executive.''
\end{enumerate}

The executive compensation restrictions required by the CPP were revised with the American Recovery and Reinvestment Act of 2009 (ARRA). Congress passed ARRA, most commonly called the `Stimulus,' in February 2009 and retroactively placed stricter limits on executive compensation. These new restrictions included rules on accrued compensation, luxury expenditures, golden parachutes, shareholder say-on-pay, and independent compensation committee, among various other restrictions. \citep{mofoComp}.

\subsection*{Application-based Capital Injections}

The CPP can be separated into the first 9 banks' participation and the application-based participation which was available to all QFIs. QFIs had until November or December 2008 to submit their application to the program, depending on what type of institution it was. Public institutions had until November 14, 2008 to apply. The application process included the following steps (as summarized by \citet{mofo}). 

\begin{enumerate}
\item After deciding to participate, an applying QFI consults with its primary federal banking regulator as it completes the application using the most recent supervisory reports available, along with any material changes in the firm's financial condition as the program was only available to ``viable'' QFIs. \citep{CPPFaq}. 
\item QFI submits the application to their primary federal banking regulator. The primary federal banking regulator then rejects or approves the application and recommends the QFI for the CPP to Treasury. 
\item Any QFI with limitations on the issuance of preferred securities or with similar limitations had to submit additional information on these limitations. 
\item Treasury's Investment Committee recommends or does not recommend investment to the Assistant Treasury Secretary for Financial Stability whose final decision accepts or rejects the QFIs application ``giving considerable weight to the recommendation of the primary federal banking regulator.'' 
\item Finally, the QFI could accept or reject the Treasury's capital investment.
\end{enumerate}

The involved regulators standardized the forms and worked to make the application and process consistent across the various federal banking regulators. Treasury did not publicly disclose its methodology for approving or disproving applications, however \citet{RejectionRates} finds the FDIC rejection rate was 11 percent, the Federal Reserve's was between 20 and 39 percent, and Treasury approved almost all application it received. There was a number of banks that also received approval -- or otherwise would have -- but withdrew from the program.\footnote{The number is not known exactly, but \citet{RejectionRates} notes 158 applications to the Federal Reserve were withdrawn, of which the Federal Reserve Office of the Inspector General said the ``majority'' were eligible to participate.} Once a QFI and Treasury agreed to a CPP investment, Treasury announced it within 48 hours. There was no public disclosure of QFIs which were rejected or withdrew from the program. 

Accounting guidance at the time of the CPP announcement required QFIs to account for the CPP warrants as bifurcated instruments or mark-to-market liabilities. To prevent the warrants from affecting the income statement in adverse ways, the SEC and Financial Accounting Standards Board (FASB) ``released guidance that, despite accounting guidance, the warrants in the [CPP] may be treated as permanent equity.'' \citep{mofo}. 

\subsection*{Unwinding CPP Investments}

As Treasury began to wind down its CPP portfolio, Treasury had three options: wait for repayment from the firm, restructure (through a merger, for example), or sell the investment via auction.\footnote{Please see Xu (2016) for further details on the mechanics of Treasury's exit from its CPP portfolio.} \citep{MassadExit}. The frequency and type of exit from the CPP is shown in Figure~\ref{exits}. 

\paragraph{Repayment}

Consistent with the characteristics of high quality capital, Treasury did not require CPP firms to repay their capital investments at any specific time frame. Instead, firms repaid when they felt appropriate. In practice, many of the largest banks repaid their CPP investments through 2009, either as a response to the various TARP limits involved with the program or because of some amount of stigma surrounding the program. As of February 2016, 261 firms had repaid their CPP investments in full.\footnote{This includes firms that refinanced \$2.21 billion in CPP investments through the Small Business Lending Fund (SBLF) and \$360 million in exchanges of CPP investments with the Community Development Capital Initiative. \citep{ProgramStatus}.}

\paragraph{Restructuring}

Firms could restructure their capital investments in connection with a merger or some other plan to raise capital. In such a transaction, Treasury would receive cash -- sometimes as a discount to the original investment -- or other marketable securities. Roughly 40 institutions restructured their CPP investments or merged with other institutions. Treasury had discretion in accepting or rejecting a restructuring offer in its effort to ensure taxpayers maximize the value of their investments.

\paragraph{Auctions}

Treasury's third option was to sell its CPP securities through auctions. Auctions sold either a single institution's CPP securities or pooled many firms' securities together depending on the size of the QFI. The bank (with their regulators' approval) or a designated bidder, normally a familiar shareholder could submit an ``opt-out bid'' to be removed from the set of firms to be auctioned. The auction used a modified Dutch auction in which the price of securities lowered until there were enough bids to sell all the securities. All the securities are then sold at that price. This is considered a``modified'' auction in the sense that there was a floor, often set by the firm's opt-out bid. Treasury previously used Dutch auctions to sell CPP warrants. 

In pooled auctions, a single bidder was allocated all auctioned securities. In single institution auctions, many bidders were allocated portions of the auctioned securities at the single clearing price. As of February 2016, Treasury led 28 auctions for 190 CPP institutions yielding \$3.04 billion in proceeds. This amounts to about 80 percent of the face value of the CPP investments. 

\subsection{Outcomes} 

By December 9, 2008, Treasury used \$204.9 billion in 742 transactions involving 707 financial institutions, less than the initial outlay of \$250 billion. Some banks turned down CPP funds after receiving approval from Treasury, and these banks had higher quality assets or were in better performing regions of the country. This suggests stronger banks viewed the CPP as costly. \citep{Ba}. As of February 2016, the status of the CPP is as follows: repayments of \$199.6 billion, write-downs of \$5.1 billion, \$300 million of outstanding investments, and \$27.1 billion of total income. In sum, Treasury recovered \$226.7 billion, as of February 2016. Figure~\ref{outcome} provides further breakdowns of the program status to date. Of the 707 financial institutions with CPP investments: 
\begin{itemize}[label={--}]
\item Full repayment: 261 
\item Sold at auction: 190 
\item Refinanced through the Small Business Lending Fund or Community Development Capital Initiative: 165 
\item Restructured through non-auction sales: 39 
\item Bankruptcy/Receivership: 32 
\item Merged with other CPP institutions: 4
\item Remain in program: 16 (as of February 2016)
\end{itemize}
\citep{GAO}
The program skewed to larger firms: the 9 largest institutions ultimately accounted for \$134.2 billion and 331 of the 707 recipients received CPP investments below \$10 million. Firms took, on average, 2.9 percent of RWA in capital suggesting that participating firms maximized the capital they could receive from the program. Although Tier 1 capital ratios increased from 10.9 percent to 13.8 percent after the CPP investments, the aggregate amount of tangible common equity fell due to mounting credit losses and write-downs. \citep{Ba}. 

Four small banks repaid their CPP investments on March 31, 2009 and were the first set of banks to repay CPP investments. The banks cited concerns about stigma associated with the program as well as TARP's compensation limitations. \citep{SmallRepay}. \citet{Ba} compile a sample of 590 publicly traded banks with annual and quarterly financial statements and information on executive compensation; they find that 95 banks had announced their intention to repay their CPP investments by November 2009. The largest firms repaid preferred investment by June and purchased warrants by August 2009. Of the 14 remaining remaining in the program, Treasury expects most of the institutions will exit through restructuring. 

It is not possible to isolate the effect of the CPP on the banking system due to the number of simultaneous programs and events, particularly over the Columbus Day Weekend. However, it is clear that the CPP's announcement on October 14 was coincided with material tightening in both Ted spreads, as seen in Figure~\ref{ted}, and large-cap bank CDS spreads, as seen in Figure~\ref{CDS}. 

However, the CPP did not resolve market concerns surrounding the underlying health of the banking system as in February 2009 U.S. policymakers embarked on a stress test of the 19 largest BHCs with a public capital backstop available to the BHCs found to have insufficient capital. The capital backstop, the Capital Assistance Program (CAP), was structured very similarly to the CPP and was similarly available to all QFIs in the US. 

The CPP and the CAP differed because the CAP came with a 9 percent dividend (rather than 5 percent ratcheting to 9 percent after 5 years), and after 7 years the CAP preferred share mandatorily converted to common equity. The CPP had no option for conversion to common. The stress test which accompanied the CAP, called the Supervisory Capital Assessment Program (SCAP), concluded in May 2009. The SCAP publicly disclosed bank specific line-by-line exposures and expected losses under a severely adverse scenario, finding 10 firms required an additional \$75 billion. Ultimately, the market viewed the SCAP as credible and sufficiently stressful and marked a turning point in the financial crisis. \citep{Bernanke} and \citep{Geithner} 9 of the 10 firms found capital privately, and the remaining firm (GMAC) received public capital through a separate capital program available to the automotive industry, the Automotive Industry Financing Program. \citep{Ross2016a}. Thus, the CAP was never used.\footnote{See \citet{Ross2016a} for additional information on the SCAP and \citet{Ross2016b} for additional information on the CAP.} Figure~\ref{CDS} shows CDS spreads over time during the CPP, CAP, and SCAP.

\newpage
\section{Key Design Decisions}

\subsection{The CPP bought equity and not assets.}

Although the TARP had been passed with the intention of purchasing bad assets in order to shore up banks' balance sheets, Secretary Paulson moved policy towards formally using equity injection in a meeting with the President on October 7. Treasury had worked to preserve the ability to inject capital in exchange for equity in the EESA legislation in order to potentially save a systemically important financial institution from failure. However, given the quickly declining market conditions, the technical challenges of setting up an asset purchase program, and the limited funds available, Treasury decided it was best to use capital injections. \citep{paulsonbook}. 

Initially, policymakers considered a program where the government matched funds raised by banks from private investors. However, as politically palatable as matching would be, it became clear that banks were unable to raise funds privately. Additionally, Treasury did not want to use common stock because of the associated voting rights. Preferred shares became the best option as the shares could be repaid regardless the price performance of the common stock, it was non-voting in most situations, and it carried a bonus for the taxpayer in the form of a dividend. 

\subsection{The CPP and the debt guarantee program were effectively available together and not separately.}

During the Columbus Day capital injections, regulators made clear that banks could not choose to participate in the guarantee program and not in the capital program: ``[i]t was a package deal, not a la carte.'' However, the joint announcement of the CPP along with the guarantee program caused some confusion as the two programs were closely related but had a variety of differences, specifically in the compensation restrictions for each program. Moreover, while regulators in the Columbus Day meeting emphasized the two programs were offered jointly, by the letter of the law ``no capital investment by a federal regulator required for the financial institutions volunteering for the guarantee program. An institution can participate in either, both or neither, depending only on eligibility.'' \citep{mofoCap}. 

\subsection{The CPP preferred shares were not convertible to common equity, unlike the CAP.}

The CPP carried a 5 percent dividend for the first five years which increased to 9 percent thereafter , whereas the CAP carried a 9 percent dividend from the start. Unlike the CPP, however, the CAP allowed conversion to common equity at any point -- at a 10 percent discount to the share price in the 20 days leading up to the CAP's announcement. It is clear this conversion option was a key component of the program, as the CAP allowed banks to exchange their CPP shares to CAP shares beyond their maximum injection in terms of percent of risk-weighted assets. \citep{GW}.

While the conversion option may have been worth the increased dividend to some banks, there was possibly an expectation of implicit convertibility in the CPP shares -- Citi converted its CPP shares in the weeks immediately following the CAP announcement. Therefore, many banks may have felt the increased dividend was not worth the explicit convertibility option.

\subsection{The CPP had no time limit on redemption.}

The CPP had no time limit on when participating firms redeemed Treasury's CPP investment. This contrasts with the CAP which required firms to redeem or convert to common equity within 7 years. Firms were required to repay the CAP ``solely with the proceeds of one or more issuances of common stock for cash.'' \citep{CAPTerms}. \citet{GW} note that smaller firms which faced a sufficiently large cost of issuing new equity would have been forced to carry a 9 percent dividend for 7 years or otherwise convert to common at a dilutive discount to existing shareholders. This may explain why the CPP was widely used and the CAP was not used. 

\subsection{Foreign financial institutions were ineligible for the CAP.}

The CAP, consistently with the SCAP, used the same definition of QFI as defined for the purposes of the CPP as unveiled in the fall of 2008. Notably, this excluded foreign institutions and U.S. branches or agencies of foreign institutions. This is largely due to the fact that foreign bank branches and agencies have no capital of their own and are subject to a different set of regulatory requirements than depository institutions in the US. Therefore, it is not possible to stress test their capital adequacy.\footnote{For further discussion of Federal Reserve regulation of foreign institutions, see \url{https://www.newyorkfed.org/aboutthefed/fedpoint/fed26.html}.}

\subsection{The CPP dividend started at 5 percent, and increased to 9 percent after 5 years.}

The Treasury considered a structure where the CPP preferred shares carried two levels of dividends, ratcheting after a certain point. Initially, Treasury policymakers considered a higher starting dividend -- between 7 or 8 percent per year. Ultimately, however, they decided to start with 5 percent as it ``was the best way to make a capital purchase program attractive to banks while giving them an incentive to pay back the government.'' Secretary of the Treasury Hank Paulson attributes this decision at least in part to a conversation with the well-known investor Warren Buffet, who suggested a lower initial dividend. \citep{paulsonbook}.

\subsection{The CPP used preferred shares and not common shares.}

Preferred shares were politically advantageous because they carried little voting rights, however they were not as loss absorbing as common equity. Additionally, Treasury had to get permission from other regulators to ensure the preferred shares would indeed count towards Tier 1 capital for all involved institutions. \citep{paulsonbook} and \citep{FedTier1}. As market analysts focused on financial institutions' TCE which excludes preferred shares, the CAP provided the explicit option to convert government capital investments to common equity in the case of further losses and credit writedowns. Therefore, the CPP's preferred shares acted more like a ``low-interest loan than true investments in their long-term health...'' \citep{Geithner}. 

\subsection{The terms of CPP investments did not vary across firms.}

Treasury decided to offer the CPP on the same terms for all involved firms, unlike other similar preceding programs. For example, \citet{Hoshi} describes Japan's March 1998 capital injection which purchased \textyen 100 billion in subordinated debt or loans with interest rates corresponding to a bank's financial health. The CPP ultimately decided to offer uniform terms to all institutions for two reasons: it would be difficult to design fair institution-specific pricing, and because policymakers wanted to avoid stigmatizing the weaker firms with more expensive capital. Further, the U.K. intervention in the days before the CPP allowed stronger firms to refrain from participation, and the equity prices of the weaker participating firms suffered from the stigma associated with the program. ``[T]he system as a whole was undercapitalized, and unless the broader shortfall was addressed, the crisis would keep migrating from the relatively weak to the relatively strong.... Recapitalizing the entire system would benefit everyone, so allowing firms to opt out and still enjoy those benefits would have been truly unfair.'' \citep{Geithner}.

\section{Evaluation}

\citet{Hoshi} compare the CPP with capital injection programs in Japan in the 1990s. In the Japanese experience, \citet{Hoshi} finds that banks may refuse capital injections. First, accepting capital injections may signal that the firms will have higher-than-expected future losses, and therefore the market would punish existing shareholders. Second, banks may refuse because the new government claims would be senior to existing equity claims. Existing shareholders would see no benefit until after the government had been repaid, and, if the bank had debt trading at a large discount, the capital injection's value would accrue to debt holders. The reduced upside to common shareholders therefore reduces their willingness to participate in such a program. The CPP's use of preferred shares were vulnerable to this situation. 

\citet{Hoshi} additionally lays out a set of lessons to be learned from the Japanese experience and compares these to the U.S. experience. They find that -- like the Japanese -- U.S. regulators were reluctant to nationalize and wind-down the least healthy banks, citing that both Bank of America and Citigroup needed large capital injections 2 months after the CPP injections.\footnote{These injections, called the Targeted Investment Program (TIP), provided \$20 billion to each Citigroup and Bank of America in December 2008. ``The progrma gave Treasury the necessary flexibility to provide additional or new funding to financial institutions that were critical to the functioning of the financial system.'' Both institutions repaid their TIP investments in full with accrued dividends, yielding a positive return of \$4.4 billion for Treasury. \citep{TIP}.} However, they note the lack of feasible resolution policies for complex financial institutions as a reason why regulators did not choose to wind down any institutions: particularly, the inability for the government to take over an institution and continue to service swap agreements. ``Had the U.S. tried to buy Citigroup and push it through bankruptcy using the existing law it would have been operating in unchartered territory.'' \citep{Hoshi}.\footnote{Further, they note that Japanese legislators explicitly passed laws which allowed for the wind-down of major financial institutions, and used it in at least two significant cases.} 

\citet{Diamond} describes that danger from leaving toxic assets on weak firms' balance sheets: fire-sales depress asset prices below fundamental valuations and distort the incentive for healthy banks to continue lending and instead compel healthy banks to horde capital to protect against the fire-sales. \citet{Hoshi} note that sufficiently well-capitalized banks can reduce the likelihood of a fire-sale as they take the other side and prevent prices from falling so far from fundamental values: ``we see the uncertainty over asset quality being intimately tied to the size of the capital shortage.'' 

\citet{Hoshi} also note that U.S. policymakers successful avoided requiring banks to provide credit to certain companies, demographics or industries.\footnote{The support for the auto industry is an exception, however, the CPP did not incentive the creation of ``financial zombie companies.''} However, in response to political concerns surrounding how banks used the CPP funds, Treasury issued a number of ``Use of Capital'' reports which asked banks to provide information for public disclosure.\footnote{See, for example, the 2009 Use of Capital Survey: \url{https://www.treasury.gov/initiatives/financial-stability/TARP-Programs/bank-investment-programs/cap/use-of-capital/Pages/2009.aspx}.} \citep{Bernanke}. 

\citet{Ba} compile a sample of 590 publicly traded banks with annual and quarterly financial statements and information on executive compensation and study which banks and under what circumstances banks were most likely to participate in the CPP. First, they find that CPP was viewed by banks as relatively costs because many of the strongest banks refrained from participating. The banks with strong capital ratios, stable funding profiles, high average asset quality and operating in better performing regions were less likely to participate in the CPP. However, weaker banks did indeed participate, suggesting the CPP managed stigma. 

Second, \citet{Ba} finds Treasury was most likely to accept applications from larger banks with greater systemic risks rather than the banks with high levels of troubled assets. In their sample of public banks, they also find many banks which received approval for participation from Treasury but ultimately decided to withdraw their application, suggesting CPP capital was viewed as relatively costly. However, banks which announced their approval for CPP funds and later decided to not use the funds experienced no significant change in equity prices; rather, the largest gains associated with the CPP came when the program was initially announced and now when a bank was specifically approved by Treasury for the funds. They also find that compensation limits related to the CPP played an important role in whether a bank used CPP funds and also on their subsequent repayment. 

Finally, \citet{Ba} suggests the CPP may have slowed the banking system's recovery because the CPP used preferred shares and the only buffer protecting the government's claim was common equity holders. Therefore, market analysts paid close attention to banks' tangible common equity as an indication of the likelihood the government would nationalize a bank. The SCAP -- the stress test conducted between February and May 2009 -- certified a bank's capital adequacy and reassured investors of the government's intentions with respect to protecting its CPP preferred investment and valuations of bank stocks accordingly responded positively in May 2009. 

\citet{Gift} find Columbus Day weekend capital injections resulted in a \$84-107 billion net benefit to taxpayer, mostly due to reductions in the probability of bankruptcy which they estimated would have destroyed roughly one-fourth of the enterprise value of the involved banks. To calculate the net benefit, they measure the difference-in-difference between the the banks' CDS rates and of GE capital -- a large non-bank financial company that did not participate in the CPP. By this measure, the Columbus Day intervention increased debt by \$119 billion, which added to the abnormal change in bank common and preferred equity, measures the enterprise value of the involved banks increased by \$128 billion. The FDIC deposit guarantee of derivative liabilities increases this to \$131 billion. Subtracting the difference of capital injection from the value of the preferred equity and warrants given to Treasury, yields a net benefit between \$84 billion and \$107 billion; even if the deadweight loss of taxation is 30 percent, the Columbus Day injections still have a positive value of between \$71 billion and 89 billion. \citet{Gift} note that because ``all the major banks were ``forced'' to participate by a very strong arm-twisting exercise by Treasury Secretary Paulson'' the first 9 banks likely did not benefit from any certification effect about the value of assets they held. Rather, they measure that most of the net benefit from two effects: first, the negative effect of uncertainty surrounding how the government would interfere with the bank's management; and second, the positive effect of the lowered likelihood of probability. 

\citet{Gift} also compares the actual CPP Columbus Day intervention with four alternatives: the original asset purchase plan, the original asset purchase program at above-market premiums, the British intervention without any debt-guarantees, and a debt-for-equity swap. They find the debt-for-equity swap the most attractive plan, as it does not require valuation of existing assets. Of the three former plans, they find the Columbus Day injection was a good balance of up-front cost and the value at risk, with much of the value of the program coming from the economical debt guarantee program. They find that equity injections without the guarantee would have been roughly twice as expensive. They find it would have taken between \$3.1 trillion and \$4.6 trillion in asset purchases to reduce CDS rates as much as the Columbus Day injections. Finally, they find the original asset purchase program revised to overpay by 20 percent would cost about \$1 trillion to reduce CDS rates as much as the actual intervention. 
\newpage
\phantomsection

\addcontentsline{toc}{section}{References}

\nocite{*}
\bibliography{\jobname}

\newpage
\section{Appendix A - List of Resources}

\subsection{Summary of Program}

\begin{itemize}
\item
\ul{Treasury Announces TARP Capital Purchase Program Description}, U.S. Treasury, October 14, 2008 -- \emph{Treasury's detailed summary and first formal announcement of the program.} \url{https://www.treasury.gov/press-center/press-releases/Pages/hp1207.aspx}

\item
\ul{Term
 Sheet for Capital Purchase Program}, U.S. Treasury -- \emph{Treasury
 document discussing terms of investments made via the CPP.} \url{https://www.treasury.gov/press-center/press-releases/Documents/document5hp1207.pdf}

 \item
\ul{Capital Purchase Program Cheat Sheet}, Morrison Foerster, 2008 -- \emph{Morrison Foerster summary of the relevant details provided in a client note.} \url{http://media.mofo.com/docs/pdf/081031CaPPCheatSheet_Private.pdf}
 
 
\end{itemize}

\subsection{Implementation Documents}
\begin{itemize}
\item
\ul{Application Guidelines for TARP Capital Purchase Program},
 U.S. Treasury, 2008 -- \emph{Treasury instructions to guide institutions through the process of applying for CPP funds.} \url{https://www.treasury.gov/initiatives/financial-stability/TARP-Programs/bank-investment-programs/cap/Documents/application-guidelines.pdf}
 \item
\ul{Process-Related FAQs for Capital Purchase Program},
 U.S. Treasury, 2008 -- \emph{Describes common questions surrounding the CPP and summarizes eligibility and the proces of applying to the relevant federal banking regulator.} \url{https://www.treasury.gov/press-center/press-releases/Documents/faqcpp.pdf}
 \item
\ul{Term
 Sheet for Capital Assistance Program}, U.S. Treasury -- \emph{Treasury
 document discussing terms of investments made via the CAP.} \url{http://www.treasury.gov/press-center/press-releases/Documents/tg40_captermsheet.pdf}

\end{itemize}

\subsection{Legal/Regulatory Guidance}

\begin{itemize}
\item
\ul{Press Release: Regarding CPP Shares and Tier 1 Capital}, Board of Governors of the Federal Reserve System, October 16, 2008 -- \emph{Federal Reserve guidance that CPP preferred shares would be included within the definition of Tier 1 capital without restriction.} \url{https://www.federalreserve.gov/newsevents/press/bcreg/20081016b.htm}

\end{itemize}

\subsection{Press Releases/Announcements}

\begin{itemize}
 \item
 \ul{Joint Statement by Treasury, Federal Reserve, and FDIC}, October 14, 2008 -- \emph{Joint statement by U.S. policymakers describing the CPP and associated guarantee program.} \url{https://www.federalreserve.gov/newsevents/press/monetary/20081014a.htm}
 \item
 \ul{SEC Halts Short Selling of Financial Stocks to Protect Investors and Markets}, September 19, 2008 -- \emph{Press releases describing the SEC's ban on shorting financials and its rationale.} \url{https://www.sec.gov/news/press/2008/2008-211.htm}
 
\end{itemize}

\subsection{Media Stories}

\begin{itemize}
\item
\ul{MUFG to renegotiate Morgan Stanley deal}, The Financial Times, October 13,
 2008 -- \emph{Article discussing Mitusibshi's negotiations in light of Morgan Stanley's collapsing stock price and sharply higher CDS spreads.} \url{http://ftalphaville.ft.com/2008/10/13/16948/mufg-to-renegotiate-morgan-stanley-deal/}
 
\item
\ul{UK to inject \textsterling 39bn into banks}, The Financial Times, October 13,
 2008 -- \emph{Article describing the UK's recapitalization plan shortly after it was unveiled.} \url{http://www.ft.com/cms/s/0/153e175e-9883-11dd-ace3-000077b07658.html}
  \item
\ul{Four Small Banks Are the First to Pay Back TARP Funds}, New York Times, March 31,
 2009 -- \emph{Article describing the first banks which repaid CPP investments.} \url{http://nyti.ms/29aFjv3} 
\item
\ul{U.S. May Convert Banks' Bailouts to Equity Share}, New York Times, April 19,
 2009 -- \emph{Article discussion the possibility of banks converting CPP shares to common equity.} \url{http://www.nytimes.com/2009/04/20/business/20bailout.html}'
 
\end{itemize}

\subsection{Key Academic Papers}

\begin{itemize}
\item
\ul{Assessing TARP},
Bayazitova, Dinara and Shivdasani, Anil, 2012 -- \emph{Paper analyzes the banks that did and did not participate in the CPP and examines market reaction and stigma associated with the program.}
\item
\ul{Fear of Fire Sales and the Credit Freeze},
Diamond, Douglas W and Rajan, Raghuram G, 2009 -- \emph{Among other things, describes the benefits of asset purchases to prevent fire-sales.}
\item
\ul{Valuing the Treasury's Capital Assistance Program},
Glasserman, Paul and Wang, Zhenyu, 2011 -- \emph{Paper
which finds CAP to be very valuable to banks, with a discussion of why banks ultimately did not participate in the program.} \url{http://papers.ssrn.com/sol3/papers.cfm?abstract_id=1525640}
\item
\ul{Leveraged Losses: Lessons from the Mortgage Market Meltdown},
Greenlaw, David and Hatzius, Jan and Kashyap, Anil K and Shin, Hyun Song, 2008 -- \emph{Paper provides contemporary estimates of subprime mortgage losses and spillover across financial markets.}
\item
\ul{Will the U.S. bank recapitalization succeed? Eight lessons from Japan},
Hoshi, Takeo and Kashyap, Anil K, 2010 -- \emph{Paper compares Japan's capital injections in the 1990s to the CPP.}
\item
\ul{The Capital Purchase Program and Subsequent Bank SEOs},
Khan, Mozaffar and Vyas, Dushyantkumar, 2015 -- \emph{Paper describing seasoned equity offerings during and after the CPP.}
\item
\ul{The International Transmission of Financial Shocks: The Case of Japan},
Peek, Joe and Rosengren, Eric S, 1997 -- \emph{Paper describes, among other issues, the affects of impaired credit intermediation on the macroeconomy in Japan's case.}
\item
\ul{How Do Banks Use Bailout Money? Optimal Capital Structure, New Equity, and the TARP},
Taliaferro, Ryan, 2009 -- \emph{Paper provides estimates of FDIC, Federal Reserve and Treasury rejection rates of CPP applications.}
\item
\ul{Paulson's Gift},
Veronesi, Pietro and Zingales, Luigi, 2010 -- \emph{Measures the net benefit of the Columbus Day intervention, compares its price tag to other similar measures, and proposes a debt for equity swap program design.}
\end{itemize}



\subsection{Reports/Assessments}

\begin{itemize}
\item
\ul{Troubled
 Asset Relief Program: Two Year Retrospective}, Office of Financial
 Stability, October 2010 -- \emph{Office of Financial Stability report
 discussing the program and its outcomes in the context of the wider
 swath of TARP.} \url{http://www.treasury.gov/press-center/news/Documents/TARP\%20Two\%20Year\%20Retrospective_10\%2005\%2010_transmittal\%20letter.pdf}

 \item
\ul{Troubled Asset Relief Program: Capital Purchase Program Largely Has Wound Down}, Government Accountability Office, 2016-- \emph{Report summarizing Treasury's efforts to wind down their CPP investments, especially their use of auctions.} \url{http://www.gao.gov/products/GAO-16-524}
 
 
 
\end{itemize}

\section{Appendix B - Road Map}

The following is a list of the key design decisions that will likely have to be made in implementing a program similar to the Capital Purchase Program (CPP), a broad-based capital injection program with standardized terms.

\subsection{Key Questions}

\begin{outline}[enumerate]

\1 Which agency or agencies have the authority and expertise to provide the capital?
\2 What is the basis of this authority?
\2 What particular elements/terms must be satisfied to fit within the authority?
\2 After designing, have all required elements been satisfied?
\1 How should a the capital injections be structured?
\2 What sort of security should the public capital be provided through?
\2 Should the government take a voting or non-voting stake?
\2 Should economic conditions worsen, can the public capital convert into common equity?
\3 If so, should the securities convert to common at a discount or at face value?
\2 Does the investment come with a dividend? If so, what is the right balance between providing capital to firms that otherwise cannot raise capital but is also sufficiently punitive that firms work to replace it with private capital quickly?
\2 Does the dividend ratchet up after a number of years to compel firms to exit?
\2 Is there mandatory conversion to common after a time period? If so, after how long?
\2 How does the taxpayer participate in the potential future profitability of the involved firms? Does the public receive warrants, for example?
\2 How does the public exit its investment? Over what time frame? 
\2 How can participating financial institutions redeem their capital injections? With cash proceeds from equity issuance only, as in the CAP? 
\1 To what extent does the government participate in managing the participating QFI?
\2 What other constraints will firms using public capital face? (E.g. executive compensation caps, restrictions on common stock dividends, buybacks and cash acquisitions, etc.)
\2 Are there sufficient authorized shares to meet the capital backstop's requirements?
\2 Does the capital injection trigger any poison pill or covenants?
\2 What is the relationship between the capital injection's preferred shares and existing preferred shares? 
\1 Which firms are eligible? 
\2 Are foreign institutions eligible?
\2 What tests are conducted to determine capital adequacy and the amount of support the public should provide? (E.g., is there a stress test?)
\2 What metric or measure should regulators target to assess capital adequacy?
\3 Should the test focus on Tier 1 capital, Tier 1 Common capital, tangible common equity, a combination of these or something else?
\4 For example, should preferred equity, goodwill and intangible assets be included in the equity component?
\4 Should the denominator be based on risk-weighted assets, tangible assets or something else?

\end{outline}

\subsection{Implementation Steps}

\begin{enumerate}

\item Develop the description of the capital injection, including legal authority, purpose, firm eligibility, a general timeline, et cetera and seek input from industry and other stakeholders.
\item If necessary, seek approval for the program, funding et cetera.
\item Produce term sheet and securities purchase agreement for the program.\footnote{CPP Example SPA: \newline \url{https://www.treasury.gov/initiatives/financial-stability/TARP-Programs/bank-investment-programs/cap/Pages/contracts.aspx}}
\item Develop application instructions for completing the documentation necessary to participate in the capital back stop.
\item Produce capital adequacy targets with which to judge applications.
\item Find institution specific capital adequacy using supervisors and firms own' capital adequacy estimates. 
\item Compare supervisors' capital adequacy estimates with firms' own estimates and reconcile differences.
\item Provide capital to firms. 

\end{enumerate}

\newpage
\section{Figures and Tables}
\begin{figure}[h]
\caption{Ted Spread}\label{ted}
\centering
\includegraphics[width=\textwidth]{ted.pdf}
\raggedright
\footnotesize Source: Federal Reserve.
\end{figure}


\begin{table}[htbp]
\setlength\LTleft\fill
\setlength\LTright{0pt}
\begin{longtable}[l]{@{\extracolsep{\fill}}@{}ll@{}ll@{}}
\caption{Columbus Day Capital Injections}\label{columbusDay}\\
\toprule
\textbf{Firm} & \textbf{CPP Investment} &\tabularnewline
\midrule
\endhead
Citigroup & \$25 billion &\tabularnewline
JP Morgan Chase & \$25 billion &\tabularnewline
Bank of America (acquiring Merrill) & \$25 billion & ~\tabularnewline
Wells Fargo (acquiring Wachovia) & \$25 billion &\tabularnewline
Goldman Sachs & \$10 billion & \tabularnewline
Morgan Stanley & \$10 billion & \tabularnewline
Bank of New York Mellon & \$3 billion &\tabularnewline
State Street & \$2 billion &\tabularnewline
\bottomrule
\textsc{Total} & \$125 billion &\tabularnewline
\bottomrule
\multicolumn{3}{l}{\footnotesize Source: U.S. Treasury.}
\end{longtable}
\end{table}

\begin{figure}[h]
\caption{Exit Types by Year}\label{exits}
\centering
\includegraphics[width=\textwidth]{Exits.pdf}
\raggedright
\footnotesize Source: Bloomberg.
\end{figure}

\begin{figure}[h]
\caption{CPP Status, June 2016}\label{outcome}
\centering
\includegraphics[width=\textwidth]{outcome.pdf}
\raggedright
\footnotesize Source: \citet{GAO}. 
\end{figure}


\begin{figure}[h]
\caption{CPP and Large Banks' CDS Spreads}\label{CDS}
\centering
\includegraphics[width=\textwidth]{CDS.pdf}
\raggedright
\footnotesize Source: Bloomberg.
\end{figure}

\end{document}
