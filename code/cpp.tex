\documentclass[12pt]{article}
\usepackage{longtable}
\usepackage{geometry}
\geometry{left=1.5 in,right=1.5 in,top=1.00in,bottom=1.0in}

%\documentclass[justified,notoc]{tufte-handout2}
%need to figure out why figure refernces are messed up

\usepackage{natbib}   % call natbib
\setcitestyle{authoryear}  % set citation style to authoryear
\bibliographystyle{plainnat} % use the plainnat instead of plain
\usepackage{booktabs}
\usepackage{amssymb, amsmath, amsfonts}
\usepackage{outlines}
\usepackage{soul}
\usepackage[gen]{eurosym}
\usepackage{graphicx}
\usepackage{enumitem}
\setlist[enumerate]{itemsep=0mm}
\setlist[itemize]{itemsep=0mm}
\usepackage{fancyhdr}
\usepackage{authblk}
\usepackage{lineno}
\usepackage[hyphens]{url}
\usepackage{hyperref}
\hypersetup{
 colorlinks = true, %Colours links instead of ugly boxes
 urlcolor  = blue, %Colour for external hyperlinks
 linkcolor = blue, %Colour of internal links
 citecolor = blue %Colour of citations
}
\pagestyle{fancy}
\usepackage{outlines}
\usepackage{caption}
%\captionsetup[table]{name=Figure}
\graphicspath{{../input/}}
\usepackage[outdir=./]{epstopdf}


\usepackage{enumitem}
\setlist[enumerate,2]{label=\roman*)}
\setlist[enumerate,3]{label=\Roman*)}
\setlist[enumerate,4]{label=\roman*)}

%\renewcommand{\familydefault}{\sfdefault}
% ------------------------------- use \citep{FAQs} to cite
\usepackage{filecontents}
\begin{filecontents}{\jobname.bib}
}

@article{Ross2016a,
  title   = {{The Supervisory Capital Assessment Program}},
  publisher = "Yale Program on Financial Stability",
  author  = "Ross, Chase P.",
  year   = "2016",
  doi    = " ",
  volume  = " ",
  journal  = "Yale Program on Financial Stability Intervention Case",
  issn   = " ",
  url    = {http://papers.ssrn.com/sol3/papers.cfm?abstract_id=2722712},
}

@article{Ross2016b,
  title   = {{The Capital Assistance Program}},
  publisher = "Yale Program on Financial Stability",
  author  = "Ross, Chase P.",
  year   = "2016",
  doi    = " ",
  volume  = " ",
  journal  = "Yale Program on Financial Stability Intervention Case",
  issn   = " ",
}


@book{Geithner,
  title={{Stress test: Reflections on financial crises}},
  author={Geithner, Timothy F},
  year={2014},
  publisher={Crown}
}

@misc{Paulson,
  title={{October 14, 2008 Speech}},
  author={Paulson, Hank},
  year={2008},
    url   = {http://ftalphaville.ft.com/2008/10/14/17036/paulsons-nine-strong-posse/},
}

@misc{CPPInstructions,
  title={{Application Guidelines for TARP Capital Purchase Program}},
  author={Treasury},
  year={2008},
    url   = {https://www.treasury.gov/initiatives/financial-stability/TARP-Programs/bank-investment-programs/cap/Documents/application-guidelines.pdf 
},
}

@book{Bernanke,
  title={{The Courage to Act: A Memoir of a Crisis and its Aftermath}},
  author={Bernanke, Ben S},
  year={2015},
  publisher={WW Norton \& Company}
}

@article{CAPTerms,
  title   = {{U.S. Treasury Releases Terms of Capital Assistance Program}},
  author = "U.S. Treasury",
  year   = "2009",
  doi    = " ",
  volume  = " ",
  journal  = " ",
  issn   = " ",
  url    = {https://www.treasury.gov/press-center/press-releases/Pages/tg40.aspx},
}


@article{Mofo,
  title={{Capital Assistance Program Public (CAP) Cheat Sheet}},
  author={{Morrison Foerster}},
  year={2009},
url    = {http://media.mofo.com/docs/pdf/090310CAPPublicCheatSheet.pdf},
}

@article{OFS,
  title={{Troubled Asset Relief Program: Two Year Retrospective}},
  author={{Office of Financial Stability}},
  year={2010},
url    = {https://www.treasury.gov/press-center/news/Documents/TARP\%20Two\%20Year\%20Retrospective_10\%2005\%2010_transmittal\%20letter.pdf},
}


@article{CPPFaq,
  title   = {{Process-Related FAQs for Capital Purchase Program}},
  author = "U.S. Treasury",
  year   = "2008",
  doi    = " ",
  volume  = " ",
  journal  = " ",
  issn   = " ",
  url    = {https://www.treasury.gov/press-center/press-releases/Documents/faqcpp.pdf},
}


@article{Andrews,
  title={{U.S. May Convert Banks’ Bailouts to Equity Share}},
  author={Andrews, Edmund L.},
  year={2009},
  publisher={The New York Times},
  url={http://www.nytimes.com/2009/04/20/business/20bailout.html?_r=1},
}


@article{kacperczyk,
  title={When safe proved risky: commercial paper during the financial crisis of 2007--2009},
  author={Kacperczyk, Marcin and Schnabl, Philipp},
  journal={The Journal of economic perspectives},
  volume={24},
  number={1},
  pages={29--50},
  year={2010},
  publisher={American Economic Association}
}

@article{Khan,
  title={The capital purchase program and subsequent bank SEOs},
  author={Khan, Mozaffar and Vyas, Dushyantkumar},
  journal={Journal of Financial Stability},
  volume={18},
  pages={91--105},
  year={2015},
  publisher={Elsevier}
}


@book{paulsonbook,
  title={On the Brink: Inside the Race to Stop the Collapse of the Global Financial System},
  author={Paulson, Henry},
  year={2010},
  publisher={Business Plus}
}

@article{Greenlaw,
  title={Leveraged Losses: Lessons from the Mortgage Market Meltdown},
  author={Greenlaw, David and Hatzius, Jan and Kashyap, Anil K and Shin, Hyun Song},
    journal={U.S. Monetary Policy Form 2008}
}

@article{IMF2010,
  title={Funding, and Systemic Liquidity},
  author={Sovereigns, IMF},
  journal={Global Financial Stability Report},
  year={2007} 
}

@book{Elliott,
  title={A primer on bank capital},
  author={Elliott, Douglas J},
  year={2010},
  publisher={Brookings Institution}
}

@book{Sorkin,
  title={Too Big to Fail},
  author={Sorkin, Andrew Ross},
  year={2009},
  publisher={Viking}
}

@article{Peek,
  title={The international transmission of financial shocks: The case of Japan},
  author={Peek, Joe and Rosengren, Eric S},
  journal={American Economic Review},
  volume={87},
  number={4},
  year={1997}
}

@article{Hoshi,
  title={Will the US bank recapitalization succeed? Eight lessons from Japan},
  author={Hoshi, Takeo and Kashyap, Anil K},
  journal={Journal of Financial Economics},
  volume={97},
  number={3},
  pages={398--417},
  year={2010},
  publisher={Elsevier}
}

@misc{shortban,
  title={{SEC Halts Short Selling of Financial Stocks to Protect Investors and Markets}},
  author={{Securities Exchange Commission}},
  year={2008},
    url   = {https://www.sec.gov/news/press/2008/2008-211.htm},
}

@misc{MUFGMS,
  title={{MUFG to renegotiate Morgan Stanley deal}},
  author={Robinson, Gwen},
  year={2008},
    publisher={The Financial Times},	
    url   = {http://ftalphaville.ft.com/2008/10/13/16948/mufg-to-renegotiate-morgan-stanley-deal/},
}

@misc{UKPlans,
  title={{UK to inject \textsterling 39bn into banks}},
  author={Larsen, Peter},
  year={2008},
    publisher={The Financial Times},	
    url   = {http://www.ft.com/cms/s/0/153e175e-9883-11dd-ace3-000077b07658.html},
}

@misc{UKActual,
  title={{UK launches \textsterling 37bn bank rescue}},
  author={Robinson, Gwen},
  year={2008},
    publisher={The Financial Times},	
    url   = {http://www.ft.com/cms/s/0/83bc2cea-98ef-11dd-9d48-000077b07658.html},
}

@misc{GSBHC,
  title={{Goldman Sachs to Become the Fourth Largest Bank Holding Company}},
  author={{Goldman Sachs}},
  year={2008},
    url   = {http://www.goldmansachs.com/media-relations/press-releases/archived/2008/bank-holding-co.html},
}

@misc{mofo,
  title={{Capital Purchase Program Cheat Sheet}},
  author={{Morrison Foerster}},
  year={2008},
    url   = {http://media.mofo.com/docs/pdf/081031CaPPCheatSheet_Private.pdf},
}

@article{CPPTerms,
  title   = {{TARP Capital Purchase Program, Senior Preferred Stock and Warrants. Summary of Senior Preferred Terms}},
  author = "U.S. Treasury",
  year   = "2008",
  doi    = " ",
  volume  = " ",
  journal  = " ",
  issn   = " ",
  url    = {https://www.treasury.gov/press-center/press-releases/Documents/document5hp1207.pdf},
}

@misc{CPPAnnouncement,
  title={{Treasury Announces TARP Capital Purchase Program Description}},
  author={{U.S. Treasury}},
  year={2008},
    url   = {https://www.treasury.gov/press-center/press-releases/Pages/hp1207.aspx},
}

@misc{mofo2,
  title={{Update to Treasury's Capital Purchase Program}},
  author={{Morrison Foerster}},
  year={2008},
    url   = {http://www.mofo.com/~/media/Files/Resources/Publications/2008/10/Update\%20to\%20Treasurys\%20Capital\%20Purchase\%20Program/Files/081021TreasuryUpdate/FileAttachment/081021TreasuryUpdate.pdf},
}

@misc{FedTier1,
  title={{Press Release: Regarding CPP Shares and Tier 1 Capital}},
  author={{Board of Governors of the Federal Reserve System}},
  year={2008},
    url   = {https://www.federalreserve.gov/newsevents/press/bcreg/20081016b.htm},
}

@article{RejectionRates,
  title={How do banks use bailout money? Optimal capital structure, new equity, and the TARP},
  author={Taliaferro, Ryan},
  journal={Optimal Capital Structure, New Equity, and the TARP (December 21, 2009)},
  year={2009}
}

@misc{mofoComp,
  title={{TARP Executive Compensation}},
  author={{Morrison Foerster}},
  year={2009},
    url   = {http://media.mofo.com/docs/pdf/090310ExecutiveCompCheatSheet.pdf},
}

@misc{MassadExit,
  title={{Winding Down TARP's Bank Programs}},
  author={Massad, Timothy},
  year={2012},
    url   = {https://www.treasury.gov/connect/blog/Pages/Winding-Down-TARPs-Bank-Programs.aspx},
}

@article{GAO,
  title={Troubled Asset Relief Program: Capital Purchase Program Largely Has Wound Down},
  author={{Government Accountability Office}},
  year={2016}
}

@misc{ProgramStatus,
  title={{Capital Purchase Program Program Status}},
  author={{U.S. Treasury}},
  year={2016},
    url   = {https://www.treasury.gov/initiatives/financial-stability/TARP-Programs/bank-investment-programs/cap/Pages/payments.aspx},
}

@article{Ba,
  title={Assessing tarp},
  author={Bayazitova, Dinara and Shivdasani, Anil},
  journal={Review of Financial Studies},
  volume={25},
  number={2},
  pages={377--407},
  year={2012},
  publisher={Soc Financial Studies}
}

@misc{SmallRepay,
  title={{Four Small Banks Are the First to Pay Back TARP Funds}},
  author={Dash, Eric},
  year={2009},
    publisher={The New York Times},	
    url   = {http://nyti.ms/29aFjv3},
}


\end{filecontents}
% -------------------------------

\begin{document}

	\lhead{}
	\rhead{}
	\renewcommand{\headrulewidth}{0.0pt}
	\renewcommand{\footrulewidth}{0.0pt}

\title{The Capital Purchase Program}
\author{Chase P. Ross\thanks{\texttt{\href{mailto:chase.ross@yale.edu}{chase.ross@yale.edu}}}}
\affil{Yale Program on Financial Stability \\ Yale School of Management}
\date{}


\maketitle

\begin{abstract}
On October 14, 2008 US policymakers announced a plan to purchase preferred shares in stressed US banks and financial institutions to ensure the US banking system had sufficient capital to withstand 
\newline
\newline
\textbf{JEL Classification}: G01, G28, G20, H12, H81
\newline
\textbf{Keywords}: crisis intervention, Capital Purchase Program, TARP,  capital requirements, systemic risk, Tier 1 Capital

\end{abstract}
\newpage
\tableofcontents
\newpage

\section{Overview}

\subsection{Background}

Financial markets became increasing volatile beginning in the summer of 2007 and sharply more so following the bankruptcy of Lehman brothers on September 15, 2008 and the subsequent runs on AIG and the Reserve Primary Fund. Combined, the run on the securitized banking system escalated in September 2008 and financial institutions hoarded cash as uncertainty about the value of banks' assets pushed interbank funding rates up. Estimates at the time \$500 billion in losses within the mortgage market through 2008 \citep{Greenlaw}.\footnote{The IMF estimated about \$700 billion in aggregate losses and writedowns within the US banking system between 2007Q2 and 2010 Q2. \citep{IMF2010}.} The provision of private credit collapsed. Emergency liquidity provision by the Federal Reserve helped viable firms finance themselves, but policymakers and market analysts were concerned banks did not have sufficient capital to absorb additional  losses. As a response, Congress passed the Emergency Economic Stabilization Act of 2008 (EESA) in early October 2008.\footnote{Congress initially tried to pass the law, with a handful of differences, in late September but the bill did not pass. The S\&P lost 8.81\% that day.} The EESA created the Toxic Asset Relief Program (TARP) with \$700 billion in funding.

Initially, policymakers designed TARP to purchase impaired assets from qualified regulated financial institutions. These purchases would prevent banks from selling their assets at fire-sale prices and using their limited capital buffers to absorb the losses. However, policymakers at the Treasury and Federal Reserve ultimately decided it was best to purchase equity directly from banks instead for three reasons: first, policymakers were concerned the process of setting up the purchasing program's logistics would take too long\footnote{The Treasury expected it would take 45 days until the program could begin its purchases. \citep{Geithner}.}; second, it was difficult to set appropriate prices for securitized assets with no market price\footnote{For example, the Federal Reserve and JPMorgan struggled for many months to negotiate appropriate prices for the pool of Bear Stearns mortgage securities JPMorgan agreed to purchase in March 2008 as part of the Maiden Lane I transaction. \citep{Geithner}.}; and third, equity was a more efficient use of the limited TARP funds than asset purchases.\footnote{For a bank with ten to one leverage, \$1 billion in equity investments is equal to \$10 billion in asset purchases.}

Capital adequacy is vital for banks to intermediate credit. \citet{Peek} shows that banks without sufficient common equity pull back from lending. When supervisors and banks are unable or unwilling to recapitalize the banking system, outcomes for a wide variety of macroeconomic indicators are dim: investment, output and unemployment all suffer. \citep{Hoshi}. Further, in the week leading up to the announcement of capital injections, banks struggled to finance their operations while interbank funding rates increased to all-time highs: the TED spread\footnote{The TED spread is the spread between short-term US Treasury bills (T) and eurodollar futures (ED).} peaked at an all-time high of 458 basis points on October 10, 2008 as shown in Figure~\ref{ted}. Equity markets reflected the stresses, as well: the week of October 6 was the worst week for the S\& P since 1933.\footnote{As an indication of the unprecedented level of uncertainty in financial markets, on Friday October 10 the Dow fell 680 points, rebounded 631 points, wiped out the rebound, then rebounded yet again 853 points but then ultimately gave up 129 points on the day. Similarly, the S\&P lost 8\% in the last hour of trading alone. \citep{paulsonbook}.} 

The week of October 6 was particularly stressful for the US banking system. Notably, the SEC had banned short selling of financial stocks beginning September 19 and expiring on Wednesday October 8. \citep{shortban}. At the same time, Morgan Stanley struggled to remain viable as it was seen as the most vulnerable bank following Lehman. On September 29, Mitsubishi UFJ agreed to purchase a 9\% of Morgan Stanley's at \$25.25, with the deal closing on October 13. Morgan Stanley executives expressed frustration that the short-sale ban would expire just days before the deal closed. After the ban expired, Morgan Stanley's shares fell more than 60\% and market analysts remained uncertainty whether the deal would indeed the following week. Ultimately, Mitsubishi renegotiated the deal to purchase a 21\% stake for \$9 billion.\footnote{Treasury Secretary Paulson, concerned Mitsubishi may pull back from the deal, sent the board of Mitsubishi a letter the earlier in the day which outlined the principles of their policy actions and that they indended to protect foreign investors, but did not explicitly mention the deal or Morgan Stanley. Within a few hours, Mitsubishi had agreed to the finalized deal. \citep{paulsonbook}.}\footnote{As the deal closed on Columbus day -- a bank holiday -- Mitsubishi was unable to wire the \$9 billion check. Morgan Stanley needed the cash on an emergency basis, so Mitsubishi gave Morgan Stanley a physical check for the \$9 billion. The check is likely the largest ever written. \citep{Sorkin}.} \citep{MUFGMS}. Further, AIG had depleted its \$85 billion bridge loan the same week and US policymakers had to set up an additional \$37.8 billion program. \citep{Geithner}. 

The UK government set the precedent for direct capital injections on October 8, 2008 when UK policymakers unveiled a \textsterling 400 billion bank assistance package which included \textsterling 25 billion to recapitalize the banking system, with an additional \textsterling 25 billion available if policymakers deemed necessary. The recapitalization plan also included the creation of a guarantee scheme of \textsterling 250 billion for new wholesale debt from banks with a plan to boost Tier 1 capital in the amount supervisors deemed appropriate. UK banks had until the end of the year to submit capital plans and their plans to raise private capital. \citep{UKPlans}. However, the panicky selling leading into the weekend of October 11 compelled UK policymakers to speed up the recapitalization process so they could announce the terms of capital injections by market open on Monday October 13. The banks, Barclays in particular, sought extra time to raise private capital and therefore avoid having the government as their largest shareholder. Shortly thereafter UK regulators announced a \textsterling 9 billion investment in RBS and Lloyds/HBOS preferred shares, giving the government a majority share of RBS and more than 40\% share of Lloyds/HBOS. Barclays agreed to forgo its dividends through 2008 while it sought to raise an additional \textsterling 10 billion privately. \citep{UKActual}.

Shortly thereafter, US policymakers announced direct capital injections with TARP funds on October 14, 2008 through the so-called Capital Purchase Program (CPP) with \$250 billion from TARP funds. 

\subsection{Program Description}

This case divides the CPP into three distinct phases: the initial round of capital injection on October 13, Columbus Day, to nine of the largest bank holding companies (BHCs), the application-based CPP investments available to other, mostly smaller, banks between October 2008 and November 2008, and Treasury's subsequent exit from its CPP investments which continues as of July 2016. 

\subsection*{Columbus Day Capital Injections}

Following the tumultuous week of October 6 -- the worst week for the S\&P since 1933, as well as notable stresses on Morgan Stanley and AIG -- regulators decided they needed to do something ``dramatic.'' \citep{paulsonbook}. Policymakers had little time to develop a plan. In addition to developing a new wholesale bank debt guarantee program, former President of the New York Federal Reserve Timothy F. Geithner noted ``Treasury and Fed officials'] other challenge that weekend was to figure out the structure of capital investments--what kind of capital, who would get it, how to price it, and so forth. This would be the most sweeping government intervention in the private markets since the 1930s, and we had two days to design it.'' \citep{Geithner}. The plan they ultimately developed was similar to one used in the UK, capital injections through preferred shares alongside a guarantee\footnote{This guarantee was the Debt Guarantee Program (DGP), itself one of two components of the Temporary Liquidity Guarantee Program (TGLP). The second component was the Transaction Account Guarantee Program (TAGP). See the Yale Program on Financial Stability 's paper on the program: [tk]. See also:  \url{https://www.fdic.gov/regulations/resources/tlgp/}.} on new wholesale bank debt; in fact, US policymakers had received a copy of the UK's capital plan beforehand and found the terms to be more punitive than the US plan called for. \citep{paulsonbook}. 

Policymakers feared stigma surrounding the CPP would prevent the firms which needed capital from using the program. The UK program was voluntary and only the weaker firms accepted capital -- Barclays, for example, refrained from participation -- and markets punished those weak firms. However, no US regulator had the ability to compel private financial institutions to accept government capital. To address this, the CPP came with a 5\% dividend for the first five years which regulators hoped was cheap enough to get all nine banks to participate. Further, to persuade the most important banks to join the program and therefore prevent the program from being stigmatized, policymakers arranged a meeting between the heads of the relevant regulatory agencies with the leaders of the most systemically important banks.  

Hank Paulson, the US Treasury Secretary, invited the leaders of the major nine U.S. banks\footnote{These banks included JP Morgan, Bank of America, Citigroup, Wells Fargo, Goldman Sachs, Morgan Stanley, Merrill Lynch (although Bank of America was acquiring it), State Street and Bank of New York Mellon.} to meet on Monday October 13. Treasury did not select the banks invited to this initial CPP meeting; consistent with the final design of the CPP which called for each banks' application to be approved and overseen by their relevant federal banking regulator (FBR), the New York Fed and the OCC selected the banks. Regulators chose the banks based on their systemic importance rather than business line -- the banks included the four largest commercial banks, three investment banks\footnote{Bank of America had  agreed to purchase Merrill Lynch on September 15, 2008 with the transaction closing in the first quarter of 2009. Additionally, both Morgan Stanley and Goldman Sachs converted from independent investment banks to bank holding companies on September 21. The two converted to BHCs for similar reasons: the ``market views oversight by the Federal Reserve and the ability to source insured bank deposits as providing a greater degree of safety and soundness... [Goldman Sachs] view[s] regulation by the Federal Reserve Board as appropriate and in the best interest of protecting and growing our franchise across our diverse range of businesses.'' \citep{GSBHC}.} and two clearing and settlement banks with systemic importance to underlying financial infrastructure.  

Some bank leaders were reluctant to take the capital, but ultimately agreed to take the capital -- even if they initially felt it was unnecessary. The CPP funds and the FDIC guarantee of new wholesale bank debt were a joint package, banks could not choose one or the other. \citep{Geithner}. At its core, the program depended on whether the strongest and most systematically important banks would join the program, and by calling the leaders of these banks together regulators were able to convince them of the value of participation. Table~\ref{columbusDay} lists the amount of capital investment in each of the nine banks as determined on Columbus Day. 

A day before the CPP's announcement on October 14, Assistant Secretary Neel Kashkari announced that Treasury would pursue a standardized program to purchase equity across a wide set of financial institutions. \citep{Ba}. Thus, leaks emerged about the nature of the meeting and markets responded positively -- the S\&P had its largest one day point increase to date. The program was formally announced the next day on October 14, 2008. The press statement announcing the program noted, ``Nine large financial institutions already have agreed to participate in this program, moving quickly and collectively to signal the importance of the program for the system. These healthy institutions have voluntarily agreed to participate on the same terms that will be available to small and medium-sized banks and thrifts across the nation.'' \citep{CPPAnnouncement}. In sum, Treasury allocated \$250 billion from TARP to the CPP, with \$125 billion allocated to the first nine banks and the remainder available through the application-based program.

The following summarizes the key features of the CPP program, which are discussed in detail below:\footnote{See \citet{mofo}.}

\begin{itemize}[label={--}]
\item Preferred investment of 1-3\% risk-weighted assets (RWA) to qualified financial institutions (QFIs).
\item 5\% dividend for first 5 years, 9\% after.
\item QFIs included US BHCs and banks; excluded foreign institutions or US branches or agencies of foreign firms.
\item Included 10-year warrants with option for Treasury to purchase amount equal to 15\% of preferred equity.
\item Various compensation and management restrictions (increased in February 2009).
\item To exit preferred equity must be redeemed in full with ``qualified equity offering'' with regulators' approval
\item After repayment of preferred, firms could purchase back warrants at fair market price.
\item If redeemed before January 2010 firms would get a discount on the warrants.
\end{itemize}

\paragraph{Eligibility}

Qualifying financial institutions (QFIs) were eligible to apply to the CAP. QFIs included BHCs, financial holding companies, insured depository institutions, and savings and loan holding companies, that were organized and operating in the United States, and deemed viable by the appropriate federal banking agency. Financial institutions controlled by foreign entities were ineligible. S corporations and mutual depository institutions were ineligible, but were eligible for another program. Public firms electing to participate must have submitted an application to their primary federal banking regulator before November 14, 2008. \citep{mofo} and \citep{CPPTerms}. The Treasury noted the program was not ``first come first served'' in that Treasury had sufficient capital to allocate from TARP for all QFIs which chose to participate. 

\paragraph{Preferred Terms}

The CPP purchased preferred shares from qualified financial institutions (QFIs) in the amount of 1 to 3 percent of risk-weighted assets or \$25 billion, whichever was less. The CPP shares were non-voting senior preferred shares except ``on matters that could adversely affect the shares.'' \citep{CPPAnnouncement}. The shares carried a 5 percent dividend for 5 years and then increased to 9\% thereafter. The preferred shares would count towards the firm's Tier 1 capital\footnote{The Federal Reserve's interim final rule from October 17, 2008 allowed without restriction the use of CPP preferred shares within Tier 1 capital. \citep{FedTier1}.}, and was senior to common equity and pari passu with existing preferred shares. The CPP shares were callable after 3 years and Treasury was able, at any time, to transfer the shares to a third party. 

Dividends were due quarterly, and if a participating institution did not pay dividends in full for more than 6 quarterly periods Treasury could appoint two directors. Once the institution paid dividends for 4 consecutive quarters Treasury would remove its directors. The shares had a liquidation preference of \$1,000 per share, or higher if necessary given the authorized preferred stock. Additionally, QFIs with CPP shares were not allowed to issue dividends for the 3 years following the CPP investment.  The CPP also required Treasury's consent before any share repurchases other than would normally occur or for benefits. In the case that QFIs had existing covenants or other limitations on issuance -- for example, anti-dilution protections on existing preferred shares --  firms had 30 days after approval of the program to handle any necessary corporate actions. \citep{mofo2}.

\paragraph{Warrants}

In addition to Treasury's purchase of preferred shares, Treasury would also receive 10-year warrants equal to 15 percent of the preferred CPP investment. The warrants would come with a strike price equal to the average price of the firms' common stock in the 20 days preceding the issuance of the CPP preferred shares. In the case shareholders withhold consent on the warrants the warrant's strike price would decrease  15 percent per 6 months up to a maximum 45 percent discount. \citep{CPPAnnouncement}. Treasury did not receive warrants if the CPP investment was less than \$50 million or if the QFI was a certified Community Development Financial Institution. \citep{mofo}. 

\paragraph{Redemption}

Participating CPP institutions must redeem the preferred CPP shares with a qualified equity offering of any Tier 1 perpetual preferred or common stock. \citep{CPPAnnouncement}. A qualifying equity offering is the sale of Tier 1 qualifying perpetual preferred or common stock for cash. Within the first 3 years of the CPP investment the proceeds of the offering be at least 25 percent or more of the CPP preferred investment, and after 3 years can be any amount at any time. Consent of the primary federal banking regulator is required before any qualifying equity offering. Repayment  If the QFI redeemed its preferred shares before December 31, 2009, Treasury reduced the number of common shares associated with its warrants in half. The QFI could purchase any other equity securities or warrants held by Treasury at fair value after the QFI had redeemed its CPP preferred shares in full. \citep{CPPTerms} and \citep{mofo2}. 

\paragraph{Executive Compensation Restrictions}

The CPP as announced four main compensation, requiring financial institutions (quoted from \citet{CPPAnnouncement}):
\begin{enumerate}
\item `` ensur[e] that incentive compensation for senior executives does not encourage unnecessary and excessive risks that threaten the value of the financial institution,''
\item ``required clawback of any bonus or incentive compensation paid to a senior executive based on statements of earnings, gains or other criteria that are later proven to be materially inaccurate,''
\item ``prohibition on the financial institution from making any golden parachute payment to a senior executive based on the Internal Revenue Code provision,''
\item ``agreement not to deduct for tax purposes executive compensation in excess of \$500,000 for each senior executive.''
\end{enumerate}

The executive compensation restrictions required by the CPP were revised with the American Recovery and Reinvestment Act of 2009 (ARRA). Congress passed ARRA, most commonly called the `Stimulus,' in February 2009 and retroactively placed stricter limits on executive compensation. These new restrictions included rules on accrued compensation, luxury expenditures, golden parachutes, shareholder say-on-pay, and independent compensation committee, among various other restrictions. \citep{mofoComp}.

\subsection*{Application-based Capital Injections}

The CPP can be separated into the first 9 banks' participation and the application-based participation which was available to all QFIs. QFIs had until November or December 2008 to submit their application to the program, depending on what type of institution it was. Public institutions had until November 14, 2008 to apply. The application process included the following steps (as summarized by \citet{mofo}). 

\begin{enumerate}
\item After deciding to participate, an applying QFI consults with its primary federal banking regulator as it completes the application using the most recent supervisory reports available, along with any material changes in the firm's financial condition as the program was only available to ``viable'' QFIs. \citep{CPPFaq}. 
\item QFI submits the application to their primary federal banking regulator. The primary federal banking regulator then rejects or approves the application and recommends the QFI for the CPP to Treasury. 
\item Any QFI with limitations on the issuance of preferred securities or with similar limitations had to submit additional information on these limitations. 
\item Treasury's Investment Committee recommends or does not recommend investment to the Assistant Treasury Secretary for Financial Stability whose final decision accepts or rejects the QFIs application ``giving considerable weight to the recommendation of the primary federal banking regulator.'' 
\item Finally, the QFI could accept or reject the Treasury's capital investment.
\end{enumerate}

The involved regulators standardized the forms and worked to make the application and process consistent across the various federal banking regulators.  Treasury did not publicly disclose its methodology for approving or disproving applications, however \citet{RejectionRates} finds the FDIC rejection rate was 11 percent, the Federal Reserve's was between 20 and 39 percent, and Treasury approved almost all application it received. There was a number of banks that also received approval -- or otherwise would have -- but withdrew from the program.\footnote{The number is not known exactly, but \citet{RejectionRates} notes 158 applications to the Federal Reserve were withdrawn, of which the Federal Reserve Office of the Inspector General said the ``majority'' were eligible to participate.} Once a QFI and Treasury agreed to a CPP investment, Treasury announced it within 48 hours. There was no public disclosure of QFIs which were rejected or withdrew from the program. 

Accounting guidance at the time of the CPP announcement required QFIs to account for the CPP warrants as bifurcated instruments or mark-to-market liabilities. To prevent the warrants from affecting the income statement in adverse ways, the SEC and Financial Accounting Standards Board (FASB) ``released guidance that, despite accounting guidance, the warrants in the [CPP] may be treated as permanent equity.'' \citep{mofo}. 

\subsection*{Unwinding CPP Investments}

As Treasury began to wind down its CPP portfolio, Treasury had three options: wait for repayment from the firm, restructure (through a merger, for example), or sell the investment via auction.\footnote{Please see Xu (2016) for further details on the mechanics of Treasury's exit from its CPP portfolio.} \citep{MassadExit}. The frequency and type of exit from the CPP is shown in Figure~\ref{exits}. 

\paragraph{Repayment}

Consistent with the characteristics of high quality capital, Treasury did not require CPP firms to repay their capital investments at any specific time frame. Instead, firms repaid when they felt appropriate. In practice, many of the largest banks repaid their CPP investments through 2009, either as a response to the various TARP limits involved with the program or because of some amount of stigma surrounding the program. As of February 2016, 261 firms had repaid their CPP investments in full.\footnote{This includes firms that refinanced \$2.21 billion in CPP investments through the Small Business Lending Fund (SBLF) and \$360 million in exchanges of CPP investments with the Community Development Capital Initiative. \citep{ProgramStatus}.}

\paragraph{Restructuring}

Firms could restructure their capital investments in connection with a merger or some other plan to raise capital. In such a transaction, Treasury would receive cash -- sometimes as a discount to the original investment -- or other marketable securities. Roughly 40 institutions restructured their CPP investments or merged with other institutions. Treasury had discretion in accepting or rejecting a restructuring offer in its effort to ensure taxpayers maximize the value of their investments.

\paragraph{Auctions}

Treasury's third option was to sell its CPP securities through auctions. Auctions sold either a single institution's CPP securities or pooled many firms' securities together depending on the size of the QFI. The bank (with their regulators' approval) or a designated bidder, normally a familiar shareholder could submit an ``opt-out bid'' to be removed from the set of firms to be auctioned. The auction used a modified Dutch auction in which the price of securities lowered until there were enough bids to sell all the securities. All the securities are then sold at that price. This is considered a``modified'' auction in the sense that there was a floor, often set by the firm's opt-out bid. Treasury previously used Dutch auctions to sell CPP warrants. 

In pooled auctions, a single bidder was allocated all auctioned securities. In single institution auctions, many bidders were allocated portions of the auctioned securities at the single clearing price. As of February 2016, Treasury led 28 auctions for 190 CPP institutions yielding \$3.04 billion in proceeds. This amounts to about 80\% of the face value of the CPP investments. 

\subsection{Outcomes} 

By December 9, 2008, Treasury used \$204.9 billion in 742 transactions involving 707 financial institutions, less than the initial outlay of \$250 billion. Some banks turned down CPP funds after receiving approval from Treasury, and these banks had higher quality assets or were in better performing regions of the country. This suggests stronger banks viewed the CPP as costly. \citep{Ba}. As of February 2016, the status of the CPP is as follows: repayments of \$199.6 billion, write-downs of \$5.1 billion, \$300 million of outstanding investments, and \$27.1 billion of total income. In sum, Treasury recovered \$226.7 billion, as of February 2016.  Figure~\ref{outcome} provides further breakdowns of the program status to date. Of the 707 financial institutions with CPP investments: 
\begin{itemize}[label={--}]
\item Full repayment: 261 
\item Sold at auction: 190 
\item Refinanced through the Small Business Lending Fund or Community Development Capital Initiative: 165 
\item Restructured through non-auction sales: 39 
\item Bankruptcy/Receivership: 32 
\item Merged with other CPP institutions: 4
\item Remain in program: 16 (as of February 2016)
\end{itemize}
\citep{GAO}
The program skewed to larger firms: the 9 largest institutions ultimately accounted for \$134.2 billion and 331 of the 707 recipients received CPP investments below \$10 million. Firms took, on average, 2.9 percent of RWA in capital suggesting that participating firms maximized the capital they could receive from the program. Although Tier 1 capital ratios increased from 10.9 percent to 13.8 percent after the CPP investments, the aggregate amount of tangible common equity fell due to mounting credit losses and write-downs. \citep{Ba}. 

Four small banks repaid their CPP investments on March 31, 2009 and were the first set of banks to repay CPP investments. The banks cited concerns about stigma associated with the program as well as TARP's compensation limitations. \citep{SmallRepay}. \citet{Ba} compile a sample of 590 publicly traded banks with annual and quarterly financial statements and information on executive compensation; they find that 95 banks had announced their intention to repay their CPP investments by November 2009. The largest firms repaid preferred investment by June and purchased warrants by August 2009. Of the 14 remaining remaining in the program, Treasury expects most of the institutions will exit through restructuring. 

It is not possible to isolate the effect of the CPP on the banking system due to the number of simultaneous programs and events, particularly over the Columbus Day Weekend. However, it is clear that the CPP's announcement on October 14 was coincided with material tightening in both Ted spreads, as seen in Figure~\ref{ted}, and large-cap bank CDS spreads, as seen in Figure~\ref{cds}. However, the CPP did not resolve market concerns surrounding the underlying health of the banking system as in February 2009 US policymakers embarked on a stress test of the 19 largest BHCs with a public capital backstop available to the BHCs found to have insufficient capital. The capital backstop, the Capital Assistance Program (CAP), was structured very similarly to the CPP and was similarly available to all QFIs in the US. The CPP and the CAP differed because the CAP came with a 9 percent dividend (rather than 5 percent ratcheting to 9 percent after 5 years), and after 7 years the CAP preferred share mandatorily converted to common equity. The CPP had no option for conversion to common. The stress test which accompanied the CAP, called the Supervisory Capital Assessment Program (SCAP), concluded in May 2009. The SCAP publicly disclosed bank specific line-by-line exposures and expected losses under a severely adverse scenario, finding 10 firms required an additional \$75 billion. Ultimately, the market viewed the SCAP as credible and sufficiently stressful and marked a turning point in the financial crisis. \citep{Bernanke} and \citep{Geithner} 9 of the 10 firms found capital privately, and the remaining firm (GMAC) received public capital through a separate capital program available to the automotive industry, the Automotive Industry Financing Program. \citep{OFR}. Thus, the CAP was never used.\footnote{See \citet{Ross2016a} for additional information on the SCAP and \citet{Ross2016b} for additional information on the CAP.}




















\section{Key Design Decisions}

\subsection{The CPP preferred shares were not convertible to common equity, unlike the CAP.}

\subsection{Foreign financial institutions were ineligible for the CAP.}

The CAP, consistently with the SCAP, used the same definition of QFI as defined for the purposes of the CPP as unveiled in the fall of 2008. Notably, this excluded foreign institutions and U.S. branches or agencies of foreign institutions. This is largely due to the fact that foreign bank branches and agencies have no capital of their own and are subject to a different set of regulatory requirements than depository institutions in the US. Therefore, it is not possible to stress test their capital adequacy. \footnote{For further discussion of Federal Reserve regulation of foreign institutions, see \url{https://www.newyorkfed.org/aboutthefed/fedpoint/fed26.html}.}

\subsection{The CPP boosted Tier 1 capital, but not tangible common equity.}

\subsection{The CPP dividend started at 5 percent, and increased to 9 percent after 5 years.}

\section{Evaluation}

Many banks turned down CPP funds after Treasury approval, so CPP funds were viewed as relatively costly.  Which banks participated in the CPP? The banks that needed the capital the most received funding and strongest banks opted out $\rightarrow$ adverse signaling appeared to be minimized. Bayazitova (2012) finds Treasury was most likely to accept applications from larger banks with greater systemic risks; Treasury did not provide capital to banks with high levels of troubled assets. 

Taliaferro (2009) estimates FDIC's rejection rate at 11\%, Fed's between 20-39\%.

Veronesi and Zingales (2009) find CPP resulted in a \$84-107 billion net benefit to taxpayer, mostly due to reductions in the probability of bankruptcy. 

CPP may have prolonged banking recovery; CPP preferred was only buffered by common equity, so we needed SCAP to certify TCE as adequate. 

\paragraph{Auctions}

Opt-out bid = floor price:  some questioned the use of the firm's opt-out bid as the auction floor; Treasury did not explicitly say they'd do this. 
Disclosure requirements: publicly traded banks complained the SEC required them to publicly disclose their intent to bid and the amount of capital raised to do so.
No matching bids: many banks wanted the ability to match the winner's bid, but Treasury didn't allow it.


\phantomsection

\addcontentsline{toc}{section}{References}

\nocite{*}
\bibliography{\jobname}

\section{Appendix A - List of Resources}

\subsection{Summary of Program}

\begin{itemize}

\item
\ul{Treasury White Paper: The Capital Assistance Program and its Role in the Financial Stability Plan}, US Treasury, February 2009 -- \emph{Treasury white paper describing how the CAP fits within the broader financial stability plan, the contingent capital framework, and the program's specific design elements.} \url{https://www.treasury.gov/press-center/press-releases/Pages/tg40.aspx}
\item
\ul{Term
  Sheet for Capital Assistance Program}, U.S. Treasury -- \emph{Treasury
  document discussing terms of investments made via the CAP.} \url{http://www.treasury.gov/press-center/press-releases/Documents/tg40_captermsheet.pdf}
\end{itemize}

\subsection{Implementation Documents}
\begin{itemize}
\item
\ul{The
  Supervisory Capital Assessment Program: Design and Implementation},
  Board of Governors of the Federal Reserve System, April 24, 2009 -- \emph{Federal Reserve document outlining design details of the SCAP.} \url{http://www.federalreserve.gov/bankinforeg/bcreg20090424a1.pdf}
\end{itemize}

\subsection{Legal/Regulatory Guidance}

\begin{itemize}
\item
\ul{Recovery Plan's Retroactive Restrictions and Say-on-Pay}, Morrison Foerster, March 2009 -- \emph{The American Recovery and Reinvestment Act of 2009 changed executive compensation restrictions on firms participating in EESA programs; this document outlines the relevant changes.} \url{http://media.mofo.com/files/uploads/Images/090302NewEra.pdf}
\end{itemize}

\subsection{Press Releases/Announcements}

\begin{itemize}
\item
\ul{U.S. Treasury Releases Terms of Capital Assistance Program}, US Treasury, February 25, 2009 -- \emph{Treasury press release describing how the Treasury and Federal banking agencies would test large bank holding companies with the SCAP and how the CAP would be used together with the SCAP.} \url{http://www.federalreserve.gov/newsevents/press/bcreg/bcreg20090507a1.pdf}
\item
\ul{ Treasury Announcement Regarding the Capital Assistance Program}, November 9, 2009 -- \emph{Press release announcing the closure of the CAP after making no investments.} \url{https://www.treasury.gov/press-center/press-releases/Pages/tg359.aspx}
\item
\ul{SCAP
  Results}, Federal Reserve, May 7, 2009 -- \emph{Press release which
  announces the results of the SCAP.} \url{http://www.federalreserve.gov/newsevents/press/bcreg/20090507a.htm}
\item
\ul{Statement
  by Timothy F. Geithner U. S. Secretary of the Treasury before the
  Senate Banking Committee}, May 20, 2009 -- \emph{Secretary Geithner
  discusses the initial impact of and market response to the SCAP's
  results.} \url{https://www.treasury.gov/press-center/press-releases/Pages/tg139.aspx}
\end{itemize}

\subsection{Media Stories}

\begin{itemize}
\item
\ul{Citigroup Sheds Energy Unit and Its \$100 Million Trader}, New York Times, October 9,
  2009 -- \emph{Article discussion sale of Philbro by Citigroup.} \url{http://www.nytimes.com/2009/10/10/business/10citi.html}
\item
\ul{U.S. May Convert Banks’ Bailouts to Equity Share}, New York Times, April 19,
  2009 -- \emph{Article discussion the possibility of banks converting CPP shares to common equity.} \url{http://www.nytimes.com/2009/04/20/business/20bailout.html}
\end{itemize}

\subsection{Key Academic Papers}

\begin{itemize}
\item
\ul{Valuing the Treasury's Capital Assistance Program},
Paul Glasserman and Zhenyu Wang, 2011 -- \emph{Paper
which finds CAP to be very valuable to banks, with a discussion of why banks ultimately did not participate in the program.} \url{http://papers.ssrn.com/sol3/papers.cfm?abstract_id=1525640}
\item
\ul{A (Mostly) Private Capital Assistance Programme
(CAP)},
Richard Caballero, 2009 -- \emph{Paper describing an alternative to the CAP which set a government guaranteed floor on bank stock prices.} \url{http://voxeu.org/article/recapitalising-banks-caballero-plan}
\end{itemize}
\subsection{Reports/Assessments}

\begin{itemize}
\item
\ul{Troubled
  Asset Relief Program: Two Year Retrospective}, Office of Financial
  Stability, October 2010 -- \emph{Office of Financial Stability report
  discussing the program and its outcomes in the context of the wider
  swath of TARP.} \url{http://www.treasury.gov/press-center/news/Documents/TARP\%20Two\%20Year\%20Retrospective_10\%2005\%2010_transmittal\%20letter.pdf}
\end{itemize}

\section{Appendix B - Road Map}

The following is a list of the key design decisions that will likely have to be made in implementing a program similar to the Capital Assistance Program (CAP), a  capital backstop program available to large bank holding companies deemed to have insufficient capital following a stress test.

\subsection{Key Questions}

\begin{outline}[enumerate]

\1 Which agency or agencies have the authority and expertise to provide the capital backstop?
\2 What is the basis of this authority?
\2 What particular elements/terms must be satisfied to fit within the authority?
\2 After designing, have all required elements been satisfied?
\2 Is any additional authority required in order to provide a capital backstop?
\2 How long should firms be allowed to seek private capital before turning to the public backstop?
\1 How should a public capital backstop be structured?
\2 What sort of security should the public capital be provided through?
\2 Should economic conditions worsen, can the public capital convert into common equity?
\3 If so, should the securities convert to common at a discount or at face value?
\2 How can the backstop be structured to compel firms to first raise private capital and use the public capital as a less preferred option?
\2 Does the backstop come with a dividend? If so, what is the right balance between providing capital to firms that otherwise cannot raise capital but is also sufficiently punitive that firms work to replace it with private capital quickly?
\2 Is there mandatory conversion to common after a time period? If so, after how long?
\2 How does the taxpayer participate in the potential future profitability of the involved firms? Does the public receive warrants, for example?
\2 How does the public exit its investment? Over what time frame? 
\2 How can participating financial institutions redeem their capital injections? With cash proceeds from equity issuance only, as in the CAP? 
\1 To what extent does the government participate ?
\2 To what extend does the public influence management decision making?
\2 What other constraints will firms using public capital face? (E.g. executive compensation caps, restrictions on common stock dividends, buybacks and cash acquisitions, etc.)
\2 Are there sufficient authorized shares to meet the capital backstop's requirements?
\2 Does the capital injection trigger any poison pill or covenants?
\2 What is the relationship between the capital injection's preferred shares and existing preferred shares? 
\1 Which firms are eligible for the capital backstop? 
\2 Are foreign institutions eligible?
\2 What tests are conducted to determine capital adequacy and the amount of support the public should provide? (E.g., is there a stress test?)
\2 What metric or measure should regulators target to assess capital adequacy?
\3 Should the test focus on Tier 1 capital, Tier 1 Common capital, tangible common equity, a combination of these or something else?
\4 For example, should preferred equity, goodwill and intangible assets be included in the equity component?
\4 Should the denominator be based on risk-weighted assets, tangible assets or something else?

\end{outline}

\subsection{Implementation Steps}

\begin{enumerate}

\item Develop the description of the capital backstop, including legal authority, purpose, firm eligibility, a general timeline, et cetera and seek input from industry and other stakeholders.
\item If necessary, seek approval for the program, funding et cetera.
\item Produce term sheet and securities purchase agreement for the program.\footnote{CPP Example SPA: \newline \url{https://www.treasury.gov/initiatives/financial-stability/TARP-Programs/bank-investment-programs/cap/Pages/contracts.aspx}}
\item Develop application instructions for completing the documentation necessary to participate in the capital back stop.
\item Produce capital adequacy targets with which to judge applications.
\item Find institution specific capital adequacy using supervisors and firms own' capital adequacy estimates. 
\item Compare supervisors' capital adequacy estimates with firms' own estimates and reconcile differences.
\item Provide capital to firms with inadequate capital. 

\end{enumerate}

\section{Figures and Tables}
\begin{figure}[h]
\caption{Ted Spread}\label{ted}
\centering
\includegraphics[width=\textwidth]{ted.pdf}
\raggedright
\footnotesize Source: Federal Reserve.
\end{figure}


\begin{table}[htbp]
\setlength\LTleft\fill
\setlength\LTright{0pt}
\begin{longtable}[l]{@{\extracolsep{\fill}}@{}ll@{}ll@{}}
\caption{Columbus Day Capital Injections}\label{columbusDay}\\
\toprule
\textbf{Firm} & \textbf{CPP Investment} &\tabularnewline
\midrule
\endhead
Citigroup & \$25 billion &\tabularnewline
JP Morgan Chase & \$25 billion &\tabularnewline
Bank of America (acquiring Merrill) & \$25 billion & ~\tabularnewline
Wells Fargo (acquiring Wachovia) & \$25 billion &\tabularnewline
Goldman Sachs & \$10 billion & \tabularnewline
Morgan Stanley & \$10 billion & \tabularnewline
Bank of New York Mellon & \$3 billion &\tabularnewline
State Street &  \$2 billion &\tabularnewline
\bottomrule
\textsc{Total} &  \$125 billion &\tabularnewline
\bottomrule
\multicolumn{3}{l}{\footnotesize Source: U.S. Treasury.}
\end{longtable}
\end{table}

\begin{figure}[h]
\caption{Exit Types by Year}\label{exits}
\centering
\includegraphics[width=\textwidth]{Exits.pdf}
\raggedright
\footnotesize Source: Bloomberg.
\end{figure}

\begin{figure}[h]
\caption{CPP Status, June 2016}\label{outcome}
\centering
\includegraphics[width=\textwidth]{outcome.pdf}
\raggedright
\footnotesize Source: \citet{GAO}. 
\end{figure}


\begin{figure}[h]
\caption{CPP and Large Banks' CDS Spreads}\label{figure1}
\centering
\includegraphics[width=\textwidth]{CDS.pdf}
\raggedright
\footnotesize Source: Bloomberg.
\end{figure}

\end{document}
