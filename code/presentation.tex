\documentclass[8pt]{beamer}
%\usetheme{CambridgeUS}
%\usecolortheme{dove}
\setbeamertemplate{caption}[numbered]
\setbeamertemplate{footline}[frame number]
\setbeamerfont{caption}{size=\scriptsize}
\usepackage[gen]{eurosym}
\usepackage{float}
\usepackage{enumitem}
\setlist[itemize]{leftmargin=*}
\usepackage{moresize}
\pdfmapfile{+sansmathaccent.map}

\usepackage{tikz}
\usepackage{pgfpages}
\setbeamertemplate{background canvas}{
    \tikz \draw (current page.north west) rectangle (current page.south east);
        }
\pgfpagesuselayout{2 on 1}[letterpaper,border shrink=5mm]
\beamertemplatenavigationsymbolsempty
\usepackage{caption}
%\setbeameroption{show notes} %un-comment to see the notes
\setbeamerfont{note page}{size=\ssmall}
\graphicspath{{../input/}}
\usepackage[outdir=./]{epstopdf}


\begin{document}

\DeclareGraphicsExtensions{.eps}

\title{Capital Purchase Program}
%\author{Chase Ross}
%\date{March 23, 2016}

%\beamersetaveragebackground{black}
%\begin{frame}
%\frametitle{}

%\end{frame}
%\beamersetaveragebackground{white}

%\frame{\titlepage}

%\frame{\frametitle{Table of contents}\tableofcontents}

\frame{\frametitle{Capital Purchase Program}

\vspace*{\fill}
\begin{figure}[h]
\noindent
\makebox[\textwidth]{\includegraphics[scale=.7 ]{CDS.pdf}}%

\end{figure}
\textbf{Design}
\begin{itemize}[label={--}]
\item Mostly voluntary program for ``healthy, viable'' banks as deemed by their applicable federal banking regulator.
\item Preferred investment of 1-3\% RWA; average of 2.9\%.
\item 5\% dividend for first 5 years, 9\% after.
\item Included 10-year warrants with option for Treasury to purchase amount equal to 15\% of preferred equity.
\item Various compensation and management restrictions (bolstered by ARRA in February 09).
\item To exit preferred equity must be redeemed in full with ``qualified equity offering'' with regulators' approval
\item After repayment of preferred, firms could purchase back warrants at fair market price.
\item If redeemed before January 2010 firms would get a discount on the warrants.
 \end{itemize}

}

\frame{\frametitle{ }
\begin{columns}[T] % align columns
\begin{column}{.49\textwidth}

\textbf{Outcomes}
\vspace{2mm}

\begin{itemize}[label={--}]
\item Many banks turned down CPP funds after Treasury approval, so CPP funds were viewed as relatively costly. 
\item By December 9, 2008, Treasury had only used \$205 billion  (of \$250 billion initial commitment) in 742 transactions involving 707 financial institutions.
\item Tier 1 ratios increased from 10.9\% to 13.8\%, although aggregate common equity fell due to further credit losses and write-downs.
\item The largest firms repaid preferred investment by June and purchased warrants by August 2009.
\item Of 707 institutions: 
\begin{itemize}
\item 260 full repayments
\item 190 auctioned investments
\item 32 bankruptcy/receivership
\item 17 remain 
\end{itemize}
\item Total recovered: \$226.7 billion


 \end{itemize}


\end{column}%
\hfill%
\begin{column}{.49\textwidth}

\textbf{Evaluation}
\vspace{2mm}

\begin{itemize}[label={--}]
\item Which banks participated in the CPP? The banks that needed the capital the most received funding and strongest banks opted out $\rightarrow$ adverse signaling appeared to be minimized.
\item Bayazitova (2012) finds Treasury was most likely to accept applications from larger banks with greater systemic risks; Treasury did not provide capital to banks with high levels of troubled assets. 
\item Taliaferro (2009) estimates FDIC's rejection rate at 11\%, Fed's between 20-39\%.
\item Veronesi and Zingales (2009) find CPP resulted in a \$84-107 billion net benefit to taxpayer, mostly due to reductions in the probability of bankruptcy. 
\item CPP may have prolonged banking recovery; CPP preferred was only buffered by common equity, so we needed SCAP to certify TCE as adequate. 
 \end{itemize}


\end{column}%
\end{columns}
}
\end{document}
