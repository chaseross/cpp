\section{Appendix B - Road Map}

The following is a list of the key design decisions that will likely have to be made in implementing a program similar to the 2008/9 FSA stress test, a' program intended to assess the capital needs of financial institutions receiving taxpayer support during a period of heightened uncertainty around potential losses in the banking system.

\subsection{Key Questions}

\begin{outline}[enumerate]

\1 Which agency or agencies have the authority and expertise to conduct the stress test?
\2 What is the basis of this authority?
\2 What particular elements/terms must be satisfied to fit within the authority?
\2 After designing, have all required elements been satisfied?
\1 What, if any, capital backstop should be available to firms which undergo the stress test?
\2 Does the existence or lack of a public capital backstop affect market views of the test's credibility?
\2 Is any additional authority required in order to provide a capital backstop?
\2 How can the backstop be structured to compel firms to first raise private capital and use the public capital as a less preferred option?
\2 How long should firms be allowed to seek private capital before turning to the public backstop?
\1 How should a public capital backstop be structured?
\2 What sort of security should the public capital be provided through?
\2 Should economic conditions worsen, can the public capital convert into common equity (at a discount)?
\2 What other constraints will firms using public capital face? (E.g. executive compensation caps, restrictions on common stock dividends, buybacks and cash acquisitions, etc.)
\1 Which firms are included in the stress test?
\2 How many firms can be credibly tested given the testing agency's resources?
\1 How transparent should the test results be? What level of granularity for estimates should be publicly available?
\2 If the issue is sovereign exposures, how should these sovereign exposures be stressed?
\1 How can the regulators ensure the test is viewed as credible?
\2 How should existing public support be incorporated into the test?
\2 What metric or measure should regulators target to assess capital adequacy?
\3 What is the target hurdle rate?
\3 What data on bank holdings and capital adequacy does the testing agency collect as part of its regular bank examination process? Is this data sufficiently granular for the test or will further data need to be collected?
\3 Should the test focus on Tier 1 capital, Tier 1 Common capital, Core Tier 1 capital, tangible common equity, a combination of these or something else?
\4 Should hybrid securities associated with pubic support be included?
\4 For example, should preferred equity, goodwill and intangible assets be included in the equity component?
\4 Should the denominator be based on risk-weighted assets, tangible assets or something else?
\1 Over what time frame is the stress test examining capital adequacy?
\2 Should the stress test measure in-the-moment or measure ``in the stress''?
\1 What economic scenarios are used to stress the firms?
\2 How do you ensure consistency in the forecast parameters across many jurisdictions?
\2 Who produces the scenarios, and at what level of geographic specificity are they produced?
\2 How many economic scenarios are used to stress the firms?
\2 How do these scenarios compare with contemporary private forecasts?

\end{outline}

\subsection{Implementation Steps}

\begin{enumerate}

\item Develop the description of the test, including legal authority, purpose, firm eligibility, a general timeline, et cetera and seek input from industry and other stakeholders.
\item If necessary, seek approval for the program, funding et cetera.
\item If a capital backstop will be used, establish a description of the backstop with legal authority, firms eligible to receive the backstop, the mechanics of the capital injection, et cetera and seek input from industry and other stakeholders. Produce term sheet for the program.
\item Produce economic scenarios with which to stress capital adequacy and distribute it to the tested firms.
\item If firm-level data at a sufficiently granular level is not available from the traditional bank examination processes, collect the relevant data from firms.
\item Draft detailed FAQs and template for published results.
\item Using the provided economic scenarios and bank portfolio data, both supervisors and firms individually produce capital adequacy estimates.
\item Compare supervisors' capital adequacy estimates with firms' own estimates and reconcile differences.
\item If necessary, develop instructions for completing the documentation necessary to participate in the capital back stop.
\item Publicly release stress test results.

\end{enumerate}
